\documentclass[12pt]{article}
\textwidth = 6.7 in
\textheight = 9.2 in
\oddsidemargin = 0.0 in
\evensidemargin = 0.0 in
\topmargin = 0.2 in
\headheight = 0.0 in
\headsep = 0.0 in
\parskip = 0.2in
\parindent = 0.0in

% ***********************************************************
% *********************** HEADER  ***************************
% ***********************************************************

\usepackage{amsmath} % AMS Math Package
\usepackage{amsthm} % Theorem Formatting
\usepackage{amssymb}	% Math symbols such as \mathbb
\usepackage{graphicx} % Allows for eps images
\usepackage{multicol} % Allows for multiple columns
\usepackage[dvips,letterpaper,margin=0.75in,bottom=0.5in]{geometry}
 % Sets margins and page size
\pagestyle{empty} % Removes page numbers
\makeatletter % Need for anything that contains an @ command 
\renewcommand{\maketitle} % Redefine maketitle to conserve space
{ \begingroup \vskip 10pt \begin{center} \large {\bf \@title}
	\vskip 10pt \large \@author \hskip 20pt \@date \end{center}
  \vskip 10pt \endgroup \setcounter{footnote}{0} }
\makeatother % End of region containing @ commands
\renewcommand{\labelenumi}{(\alph{enumi})} % Use letters for enumerate
% \DeclareMathOperator{\Sample}{Sample}
\let\vaccent=\v % rename builtin command \v{} to \vaccent{}
\renewcommand{\v}[1]{\ensuremath{\mathbf{#1}}} % for vectors
\newcommand{\gv}[1]{\ensuremath{\mbox{\boldmath$ #1 $}}} 
% for vectors of Greek letters
\newcommand{\uv}[1]{\ensuremath{\mathbf{\hat{#1}}}} % for unit vector
\newcommand{\abs}[1]{\left| #1 \right|} % for absolute value
\newcommand{\avg}[1]{\left< #1 \right>} % for average
\let\underdot=\d % rename builtin command \d{} to \underdot{}
\renewcommand{\d}[2]{\frac{d #1}{d #2}} % for derivatives
\newcommand{\dd}[2]{\frac{d^2 #1}{d #2^2}} % for double derivatives
\newcommand{\pd}[2]{\frac{\partial #1}{\partial #2}} 
% for partial derivatives
\newcommand{\pdd}[2]{\frac{\partial^2 #1}{\partial #2^2}} 
% for double partial derivatives
\newcommand{\pdc}[3]{\left( \frac{\partial #1}{\partial #2}
 \right)_{#3}} % for thermodynamic partial derivatives
\newcommand{\ket}[1]{\left| #1 \right>} % for Dirac bras
\newcommand{\bra}[1]{\left< #1 \right|} % for Dirac kets
\newcommand{\braket}[2]{\left< #1 \vphantom{#2} \right|
 \left. #2 \vphantom{#1} \right>} % for Dirac brackets
\newcommand{\matrixel}[3]{\left< #1 \vphantom{#2#3} \right|
 #2 \left| #3 \vphantom{#1#2} \right>} % for Dirac matrix elements
\newcommand{\grad}[1]{\gv{\nabla} #1} % for gradient
\let\divsymb=\div % rename builtin command \div to \divsymb
\renewcommand{\div}[1]{\gv{\nabla} \cdot #1} % for divergence
\newcommand{\curl}[1]{\gv{\nabla} \times #1} % for curl
\let\baraccent=\= % rename builtin command \= to \baraccent
\renewcommand{\=}[1]{\stackrel{#1}{=}} % for putting numbers above =
\newtheorem{prop}{Proposition}
\newtheorem{thm}{Theorem}[section]
\newtheorem{lem}[thm]{Lemma}
\theoremstyle{definition}
\newtheorem{dfn}{Definition}
\theoremstyle{remark}
\newtheorem*{rmk}{Remark}
% \def\rmdefault{bch} % Use Charter for main text font.

% ***********************************************************
% ********************** END HEADER *************************
% ***********************************************************
% =========================================================
\begin{document}

\begin{center}
  {\LARGE
    \textbf{CS231 Review}\\
  }
  \begin{enumerate}
  \item 
    Logic
    \begin{enumerate}
    \item TODO
    \end{enumerate}

    % TODO
    % ----------------------------------------------------------------------------
  \item Proofs
    \begin{enumerate}
    \item Even: An integer $n$ is even if there is an integer $m$ such
      that $n = 2m$.
    \item Odd: An integer $n$ is odd if there is an integer $m$ s.t. $n =
      2m + 1$
    \item Multiplicative inverse: If $q$ and $r$ are real numbers, $r$ is a multiplicative inverse for $q$
      if $qr = 1$.
    \item Proof Methods
      \begin{tabular}{ c | c | c  }
        & prove & disprove\\
        \hline
        universal & general argument & specific counter-example\\
        existential &  specific example & general argument\\
      \end{tabular}
\end{enumerate}

% ----------------------------------------------------------------------------
\item 
  Number Theory
  \begin{enumerate}
  \item Factors and multiples\\
    $a|b \textrm{ iff } b = an$
  \item Prime Numbers\\
    an integer $q \geq 2$ is prime iff positive factors of $q$ are
    $q$ and $1$. An integer $q \geq 2$ is composite if it is not prime.
  \item GCD:\\
    Euclidean algorithm
    \begin{enumerate}
    \item def $\gcd(a,b)$:
    \item $r =$ remainder$(a,b)$
    \item if$(r = 0)$ return  $b$
    \item return $\gcd(b,r)$
    \end{enumerate}
  \item LCD: $ \frac{ab}{\gcd(a, b)}$
  \item Congruence mod k\\
    Definition: If $k$ is any positive integer, two integers $a$ and $b$, $a ≡ b \mod(k)$ if $k | (a − b)$.
  \item Equivalence classes\\
    A set of inputs that evaulate to the same output\\ 
  \end{enumerate}
  \newpage
  % ----------------------------------------------------------------------------
  \item Sets:\\
    \begin{enumerate}
    \item Sets: Definition: A set is an unordered collection of
      objects.\\
      $\{a,b\}$, empty set = $\emptyset$
    \item Tuples:  Definition: A set is an ordered collection of
      objects.\\
      Example $(a,b)$
    \item Cardinality: Definition: the cardinality of a set is a
      measure of the "number of elements of the set"\\
      Cardinality of set $S$ = $|S|$
    \item Inclusion: Definition: $A \subseteq B$ iff $\{\forall x, x \in
      A \rightarrow x \in B\}$
    \item Set operations
      \begin{enumerate}
      \item Intersection: $A \cap B = \{S | S \in A \wedge S \in B\}$\\
        Disjoint: $A \cap B = \emptyset$
      \item Union: $A \cup B = \{x:x\in A \vee x \in B\}$
      \item Difference: $A-B = \{x:x\in A \wedge x\notin B\}$ 
      \item Complement $\bar A = \{x: x\notin A \wedge x\in U\}$
      \item Cartesian product: $A \times B = \{(x, y) | x \in A \wedge
        y \in B\}$
      \end{enumerate}
    \item DeMorgan’s Law: $\bar{A \cup B} = \bar A \cap \bar B$
    \item Inclusion-Exclusion Principle: $|A \cup B| = |A| + |B| − |A
      \cap B|$
    \item Product Rule: $|A \times B| = |A||B|$.
    \end{enumerate}
    % ----------------------------------------------------------------------------
  \item 
    Relations
    \begin{enumerate}
    \item Relation: $A$ relation $R$ on a set $A$ is a subset of $A \times A$, i.e. $R$ is a set of ordered pairs
      of elements from $A$.
    \item Reflexive: $R$ is reflexive if $xRx$ for all $x \in A$
    \item Irreflexive $R$ is irreflexive if $x \textrm{ not R }x$ for
      all $x \in A$.
    \item Symmetric: $\forall x, y \in A, xRy \rightarrow yRx$
    \item Antisymmetric: $\forall x, y \in A,(xRy \wedge yRx) \rightarrow x = y$
    \item Transitive: $\forall a, b, c \in A$, if $aRb$ and $bRc$,
      then $aRc$
    \item Types of relations:
      \begin{enumerate}
      \item Partial order: a relation that is reflexive, antisymmetric, and
        transitive.
      \item Linear order (total order): partial
        order in which every pair of elements are comparable. That is,
        for every pair $x, y$, $x \geq y$ or $y \geq x$.
      \item Strict partial order: a relation that is irreflexive,
        antisymmetric, and transitive.
      \item Equivalence relation: a relation that is reflexive, symmetric, and transitive.
      \end{enumerate}
    \end{enumerate}
    \newpage
%----------------------------------------------------------------------------
  \item 
    Functions
    \begin{enumerate}
    \item Function: For each input value, a function must provide one and only one output value.
    \item 
      Image: $f(A) = {f(x) : x \in A}$
    \item 
      Onto(surjective): $\forall y \in  B, \exists x \in A, f(x) = y$
    \item 
      One-to-One(injective): $\forall x, y \in A, f (x) = f (y) \rightarrow x =
      y$
    \item 
      Bijection(one-to-one correspondence): Onto and One-to-One
    \item 
      Pigeonhole principle: Suppose you have $n$ objects and assign $k$
      labels to these objects. If $n > k$, then two objects must get the
      same label.
    \end{enumerate}
    % ----------------------------------------------------------------------------
  \item 
    Counting
    \begin{enumerate}
    \item Permuation: $\frac{n!}{(n - k)!}$
    \item Combination: $\dbinom{n}{k}=\frac{n!}{k!(n-k)!}$
    \item Combinations with repetition:  $\dbinom{k+n-1}{k}$
    \item General Idenity: $\dbinom{n}{k} =  \dbinom{n}{n-k}$
    \item Pascals Idenity:  $\dbinom{n+1}{k} =  \dbinom{n}{k} +
      \dbinom{n}{k-1}$
    \item Binomial Theorem: $(x+y)^n=\sum_{k=0}^n\dbinom{n}{k}x^{n-k}y^k$
    \end{enumerate}
    % ----------------------------------------------------------------------------
  \item 
    Graphs
    \begin{enumerate}
    \item Simple Graph: A graph with neither multiple edges nor loop
      edges
    \item An isomorphism from $G_1$ to $G_2$ is a bijection $f : V1 \rightarrow V2$ such
      that vertices $a$ and $b$ are joined by an edge iff $f (a)$
      and $f (b)$ are joined by an edge.
    \item Complete graphs: $K_n$ is a graph with $n$ vertices in which
      every vertex is connected to every other
      vertex. $(\frac{n(n-1)}{2})$ edges
    \item Cycles: Suppose that we have n vertices named $v_1 , . . . ,
      v_n$ , where $n \geq 3$. Then the cycle $C_n$ is the graph with these
      vertices and edges connecting $v_i$ to $v_i+1$ , plus an
      additional edge from $v_n$ to $v_1$. $n$ edges
    \item Wheels: The wheel $W_n$ is just like the cycle $C_n$ except
      that it has an additional central “hub” node which is connected
      to all the others. Notice that $W_n$ has $n + 1$ nodes (not $n$ nodes). It has $2n$ edges.
    \item Walk: a walk of length $k$ from vertex $a$ to vertex $b$ is a
      finite sequence of vertices $a = v1 , v2 , . . . , v_n = b$ and $a$
      finite sequence of edges $e_1 , e_2 , . . . , e_{n−1}$ in which $e_i$ connects $v_i$ and $v_i+1$ , for all $i$
    \item Path: A path is a walk (necessarily an open walk) in which no vertex is used more than once.
    \item Acyclic: Doesn't contain any cycle
    \item Cut edge: One edge is keeping the graph as one connected component
    \item Euler Circuit: An Euler circuit of a graph $G$ is a closed
      walk that uses each edge of the graph exactly once.
    \item Bipartite: A graph $G = (V, E)$ is bipartite if we can split $V$
      into two non-overlapping subsets $V_1$ and $V_2$ such that every edge in $G$ connects an element of $V_1$ with an element of $V_2$ .
    \item HyperCube:
      \begin{enumerate}
      \item $Q_0$ is a single vertex with no edges 
      \item $Q_n$ consists of two copies of $Q_{n−1}$ with edges joining corresponding
        vertices.
      \end{enumerate}
    \end{enumerate}
    % ----------------------------------------------------------------------------
  \item 
    Induction
    \begin{enumerate}
    \item Induciton: Proof by recursion
    \item Strong Induction: Using multiple basecases to "reach back"
      further
    \end{enumerate}
    % ----------------------------------------------------------------------------
  \item Trees
    \begin{enumerate}
    \item TODO
    \end{enumerate}
    % ----------------------------------------------------------------------------
  \item 
    Big-O
    \begin{enumerate}
    \item Order of functions: $1 \prec \log{n} \prec n \prec n\log{n}\prec n^k \prec k^n \prec n!$
    \item $f(x)$ is  $O(g(X)$ iff there are positive real numbers $c$
      and $k$ s.t. $0 \leq f (x) \leq cg(x)$ for every $x \geq k$.
    \item $g(x)$ is $\Omega(f (x))$ iff $f (x)$ is $O(g(x))$
    \item $f (x)$ is $\Theta(g(x))$ iff $g(x)$ is $O(f (x))$ and $f (x)$ is $O(g(x))$
    \end{enumerate}
    \newpage
    % ----------------------------------------------------------------------------
  \item 
    Algorithms
    \begin{enumerate}
    \item Unrolling:
    \item Linear Search: $O(n)$
    \item Binary search: $O(\log{n})$
    \item Bubble sort: $O(n^2)$
    \item Insertion sort: $O(n^2)$
    \item Linked Lists: 
    \item Merge: $O(n)$
    \item Mergesort: $O(n\log{n})$
    \item Tower of Hanoi: $O(2^n)$
    \item Anatolii Karatsuba's Multiplication: $O(n^{\log_2 3} )$
    \end{enumerate}
  % ----------------------------------------------------------------------------
  \item 
    Sets of sets  
    \begin{enumerate}
    \item Powersets($\mathbb{P}$): set of all subsets of $S$, including the empty set and $S$ itself.
    \item Partitions: A set $A$ is a collection of non-empty subsets
      of $A$ which cover all the elements of $A$ and which don’t
      overlap.
      \begin{enumerate}
      \item covers all of $A$: $\cup_{x \in p}$ $X=A$
      \item non-empty: $A_i \neq \emptyset$ for all $X \in P$ 
      \item no overlap: $X \cap Y = \emptyset$ for all $X, Y \in P,X \neq Y$
      \end{enumerate}
    \end{enumerate}
  \item
    Countabillity
    \begin{enumerate}
    \item Two sets $A$ and $B$ have the same cardinality $(|A| =
      |B|)$ iff there is a bijection from $A$ to $B$
    \item An infinite set $A$ is countable iff there is a bijection from $\mathbb{N}$ onto $A$
    \item Cantor Schroeder Bernstein Theorem: Definition: $|A| \leq |B|$ iff
      there is a one-to-one function from $A$ to $B$
    \end{enumerate}
  \item 
    Planargraphs
    \begin{enumerate}
    \item Planargraph:
    \item Colorability: Always $4$ colorable
    \item Homeomorphic: One graph is a subdivision of another, or they
      are both subdivisions of some third graph.
    \item $K_{3,3}$ is not planar
    \item $K_5$ is not planar
    \item Kuratowski’s Theorem: A graph is nonplanar iff it contains a subgraph homeomorphic to $K_{3,3}$ or $K_5$ .
    \item Euler’s formula: $v − e + f = 2$
    \item Handshaking theorem (degrees): sum of vertex degrees$= 2e$
    \item Handshaking theorem (faces): sum of face degrees $= 2e$
    \end{enumerate}
  \end{enumerate}
  
\end{center}
\end{document}