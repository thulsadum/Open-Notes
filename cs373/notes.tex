% Created 2012-06-10 Sun 04:51
\documentclass[11pt]{article}
\usepackage[utf8]{inputenc}
\usepackage[T1]{fontenc}
\usepackage{graphicx}
\usepackage{longtable}
\usepackage{float}
\usepackage{wrapfig}
\usepackage{soul}
\usepackage{amssymb}
\usepackage{hyperref}
\usepackage{header}

\title{CS 373 Notes}
\author{}
\date{10 June 2012}

\begin{document}

\maketitle

\setcounter{tocdepth}{3}
\tableofcontents
\vspace*{1cm}
\section{General}
\label{sec-1}



\begin{center}
\begin{tabular}{lll}
 Sizes                 &  Examples                           &  Countable?  \\
\hline
 Finite                &  \{a,b\}                            &  yes         \\
 Countable Infinite    &  \mathbb{N},\mathbb{Z}, \mathbb{Q}  &  yes         \\
 Uncountable Infinity  &  \mathbb{R}, Pow(\mathbb{R})        &  no          \\
\end{tabular}
\end{center}





\begin{center}
\begin{tabular}{lll}
 name                   &  descriptipn  &  Machine  \\
\hline
 regular                &  LRk          &  D PDA    \\
 context free language  &  CFG          &  PDA      \\
\end{tabular}
\end{center}



\subsection{Chomsky Heirarchy}
\label{sec-1.1}


\begin{center}
\begin{tabular}{rlll}
        Type  &  Grammar            &  Rules                                                        &  Machine                          \\
\hline
           0  &  Unrestricted       &  $\alpha$ \rightarrow $\beta$                                 &  Turning Machines (recognizable)  \\
           1  &  Context-Sensitive  &  $\alpha$ \rightarrow $\beta$, \abs{\beta} \geq \abs{\alpha}  &  non-det LBA                      \\
           2  &  Context-Free       &  A \rightarrow $\alpha$                                       &  non-det PDA                      \\
 3$^{\star}$  &  Regular            &  A \rightarrow a, A \rightarrow aB                            &  DFA/NFA                          \\
\end{tabular}
\end{center}



\begin{itemize}
\item S \rightarrow $\epsilon$ also allowed, but S then cannot appear on the right side of rules
\end{itemize}
\subsection{Starting off}
\label{sec-1.2}

\begin{enumerate}
\item Alphabet($\Sigma$) = finite non empty set
\item $\mathbb{N}$ in this class starts at 0
\item A set X is countably infinite iff $\exists$ a bijection $f:\mathbb{N} \rightarrow X$
\end{enumerate}
\subsection{Strings}
\label{sec-1.3}

\begin{enumerate}
\item String(w) = sequence of characted in $\Sigma$
\item $w:\{c_i\in\Sigma |0\leq i\leq n\}$
\item $|w|= n \equiv$ length of the string
\item $|w| = 0 \rightarrow w = \epsilon$

\begin{enumerate}
\item Careful $\sigma \neq \emptyset$
\end{enumerate}

\item Substring subsequence of characters in w
\item Concatination: $w_1 \cdot w_2$
\item Reverse: $w^r$
\item Palindrum: $w = w^r$
\end{enumerate}
\subsection{Language}
\label{sec-1.4}

\begin{enumerate}
\item Language(L) = set of strings
\item $\Sigma^n = \{w : |w| = n\}$
\item $\Sigma^0 = \{ \epsilon \}$
\item $\Sigma^{\star} = \cup^n_{i=0} \Sigma^i$, Language of all strings
\end{enumerate}
\section{Regular Languages}
\label{sec-2}

\subsection{Deterministic Finite Automotas}
\label{sec-2.1}

\begin{enumerate}
\item Finite state machine (M)
\item Takes a string of inputs
\item 2 types of states

\begin{enumerate}
\item Accept
\item Deny
\end{enumerate}

\item There is 1 start state
\item The set of all strings accepted by language of M or L(A)
\item Formal Definition

\begin{enumerate}
\item a Language A $\in \Sigma^{\star}$ is called regular iff there
        exists a DFA ,M, s.t. $L(M) = A$
\item A DFA is a 5 tuple $M = (Q,\Sigma, \delta, q_0, F)$

\begin{enumerate}
\item Q is a finte set of states
\item $\Sigma$ is a finite alphabet
\item $\delta: Q\times\Sigma \rightarrow Q$ is the transition function
\item $q_0 \in Q$  is the inital state
\item $F \subset Q$ is the set of accept states
\end{enumerate}

\item L(M) $\equiv$ language of all accepted strings
\end{enumerate}

\item Closure properties/Regular Operations on languages

\begin{enumerate}
\item A$_1$ and A$_2$ are regular
\item Union: $A_1 \cup A_2 = A_3$
\item Concatenate: $A_1 A_2 = A_3$
\item Star: $A_1^{\star} = A_3$
\end{enumerate}

\end{enumerate}
\subsection{Non-Deterministic Finite Automotas (NFA)}
\label{sec-2.2}

\begin{enumerate}
\item Formal Definition

\begin{enumerate}
\item $M = (Q,\Sigma,\delta,q_0,F)$

\begin{enumerate}
\item Q = finite set of states
\item $\Sigma$  is a finate alphabet
\item $\delta = Q \times \Sigma_\epsilon \rightarrow Pow(Q)$

\begin{enumerate}
\item $\Sigma_\epsilon = \Sigma \cup {\epsilon}$
\end{enumerate}

\item $q_0$ = start state
\item $F \subset Q$
\end{enumerate}

\item NFA accepts w If we can write $w = {y_1 y_2 ... y_n
         y_i \in \Sigma_\epsilon}$ s.t. there exists a sequence of
         states path $R = {r_0,r_1,...}$

\begin{enumerate}
\item r$_0$ = q$_0$
\item $r_{i+1}\in\delta(r_i,y_{i+1})$ for $i=0,1...m-1$
\item $r_m \in F$
\end{enumerate}

\end{enumerate}

\item Useful Lemma: For all NFA, M, there exists an DFA ,N, s.t. L(m)
      = L(n)
\end{enumerate}
\subsection{Regular Expressions}
\label{sec-2.3}

\begin{enumerate}
\item Def: R is a regex over a fixed alphabet iff one of the
      following is true:

\begin{enumerate}
\item $R=a\in\Sigma$
\item $R=\sigma$
\item $R=\emptyset$
\item $R=R_1\cup R_2$, given $R_1$ and $R_2$ are regex
\item $R = R_1 R_2$, given $R_1 \wedge R_2$ are regex
\item $R = R^{\star}$
\end{enumerate}

\item Order of operations

\begin{enumerate}
\item star
\item concatination
\item union
\end{enumerate}

\item Identities

\begin{enumerate}
\item $a \emptyset = \emptyset$
\item $a \epsilon = a$
\item $\emptyset^{\star} = \{\epsilon\}$
\end{enumerate}

\end{enumerate}
\subsection{Generalized NFA (GNFA)}
\label{sec-2.4}

\begin{enumerate}
\item Definition

\begin{enumerate}
\item Q = set of all states
\item $Q^0 = Q - {q_{start}, q_{accept}}$
\item \underline{The} start state has out edges to every $q\in Q - {q_{start}}$, and
         no in edges
\item \underline{The} accept state has inedges from every $q\in Q - {q_{accept}}$,
         and no outedges
\item An edege exists from every $q_1\in Q^0$ to every $q_2\in Q^0$
         even if $q_1=q_2$
\item Every edge is labed with a regex
\end{enumerate}

\item Useful lemma: Any NFA can be written as a GNFA
\item lemma: Given a GNFA, M,  with 2 states, the regex between the 2
      states describes the language of M
\end{enumerate}
\subsection{Pumping Lemma for regular languages}
\label{sec-2.5}

  If A is regular, then $\exists p \in \mathbb{N}$ s.t. $\forall s
   \in A$ for which $|s| >= p, s$ can be written as  $xyz$ and
   satisfy the following conidition:
\begin{enumerate}
\item $\forall i >= 0, x y^i z \in A$
\item $|y|>0$ i.e. $y \neq \epsilon$
\item $|xy| <= p$
\end{enumerate}
  P is called the ``pumping length

\subsection{Subsitutions}
\label{sec-2.6}

\subsubsection{Substution simple definition}
\label{sec-2.6.1}

\begin{itemize}
\item A is a reg language and $A \mapsto f(A), A \subseteq \Sigma^{\star}$
\item A is described w. a regex and $R_a$ is a regex using $\Gamma$
\item $\forall a \in \Sigma a \mapsto R_a$
\item $\epsilon \mapsto \epsilon$ and $\emptyset \mapsto \emptyset$
\end{itemize}
\subsubsection{Homomorphism}
\label{sec-2.6.2}

\begin{itemize}
\item $A \mapsto h(A)$
\item $a \mapsto w, w \in \Gamma^*, a\in \Sigma$
\end{itemize}
\subsubsection{Inverse Homorphism}
\label{sec-2.6.3}

\begin{itemize}
\item $h^{-1}(A) = \{w\ \in \Sigma^* | h(w) \in A\}$
\end{itemize}
\subsection{DFA Minimization}
\label{sec-2.7}

\subsubsection{Theory}
\label{sec-2.7.1}

   Problem: Given a DFA, M, with L(M) = A, find another DFA, M$_{\mathrm{2c}}$,
   s.t. L(M) = L(M$_2$) and |Q$_2$| is as small as possible\\
\begin{itemize}
\item $\delta: Q \times \Sigma \rightarrow Q$
\item $\bar{\delta}(q,w) q\in Q, w\in \Sigma^{\star}$
\item $\bar{\delta}(q,w) \equiv$ interative call on delta for all $w_i$
     in w
\item If $\exists w \in \Sigma^*$ s.t. $[\bar{\delta(p,w)} \in F$ and
     $\delta{q,w} \notin F]$ or $[\bar{\delta}(p,w) \notin F$ and $\delta{q,w} \in F]$ then p and q are distrguishable
\end{itemize}
\subsubsection{Algorithm}
\label{sec-2.7.2}


\begin{verbatim}
for (p,q) in Q^2:
    if (p in F) and (not q in F):
        A.push((p,q)) # marked list
    else:
        B.push((p,q)) # unmarked list
for (p,q) in B:
    if (delta(p,a),delta(q,a)) in B:
        A.push((p,q))
\end{verbatim}


\subsection{Reg Operations (closed under the Reg languages)}
\label{sec-2.8}

\begin{enumerate}
\item $A_1 \cup A_2$
\item $A_1 - A_2$
\item $\bar{A_1} = \Sigma^{\star} - A_1$
\item $A_1 \cap A_2$
\item Symmetric Diff
\item $A_1 A_2$
\item $A_1^{\star}$
\item $A^r$
\item Reg langagues are clased under subsitution
\item Reg langagues are clased under homomorphism
\item Reg langagues are clased under inverse homorphism
\item Reg langagues are clased under
\end{enumerate}
\subsection{Right Invariant Equivilance}
\label{sec-2.9}

   Def: An equivilance relation is called \underline{right invariant} or
   \underline{concatenation invariant} iff $x\sim y \implies \forall w \in
   \Sigma^{\star} xw\sim yw$
\subsection{Myhill-Nerod Theorem}
\label{sec-2.10}

\begin{itemize}
\item \underline{Claim}: The following statments are equivalent

\begin{enumerate}
\item A is a regular language
\item $\exists$ a right invariant equivilance relation that has a
      finite index, and $A=$ union of some of the equivilance classes
\item $\stackrel{A}{\sim}$ is of finite index
\end{enumerate}

\end{itemize}
   
\begin{itemize}
\item \underline{Proof}: $1\rightarrow 2$

\begin{enumerate}
\item Let M be any DFA, $M=(Q,\Sigma,\delta,q_0,F)$ s.t. $A=L(M)$
\item Let $\stackrel{A}{\sim}$ be define as in Right Invariant Equivilance
\item The number of equivilance classes is $\leq |Q|$
\item A is then the union of classes that correspond to F $\square$
\end{enumerate}

\item \underline{Proof}: $2\rightarrow 3$

\begin{enumerate}
\item Show that the partition of $\Sigma^{\star}$ produced by a right
      invariant is a refinement of the parition induced by $\stackrel{A}{\sim}$
\item Since $\sim$ is right invariant, $\forall z\in
       \Sigma^{\star},xz\sim yz$ which implies that $xz\in A$ iff $yz
       \in A$ which by definition implies $x\stackrel{A}{\sim} y$
\end{enumerate}

\item \underline{Proof}: $3\rightarrow 1$

\begin{enumerate}
\item Construct a DFA using $\stackrel{A}{\sim}$
\item Let $Q \equiv$ set of equivilance classes of $\stackrel{A}{\sim}$
\item Let $[x]\in \Sigma^{\star}, [x]\in Q$, denote the equivilance
       class that x belongs to
\item Let $\delta([x],a)\equiv [xa]$, by def of $\stackrel{A}{\sim}$
\item Let $q_0 \equiv [\epsilon]$ and $F\equiv{[x] : x\in A}$
\end{enumerate}

\end{itemize}
\section{Context Free Langagues}
\label{sec-3}

\subsection{Formal Definition:}
\label{sec-3.1}

\begin{enumerate}
\item $(V,\Sigma, R, S)$

\begin{enumerate}
\item $V =$ Finite set of \underline{variables} or ``non-terminals''
\item $\Sigma$ =  finite set of \underline{terminals}

\begin{enumerate}
\item $\Sigma \cap V = \emptyset$
\item Convention: Variables are uppercase, symbols are lowercase
\end{enumerate}

\item R = finite set of \underline{rules} or ``subsitution rules'' or ``productions''
         1.Rules: examples

\begin{enumerate}
\item $A \rightarrow aaBc | a$

\begin{enumerate}
\item This means the for an A you can replace it with aaBc or a
\end{enumerate}

\item $A \Rightarrow OA1 \Rightarrow 00A11 \Rightarrow 001A011 \Rightarrow 001011$
\end{enumerate}

\item S is the \underline{start variable}
\end{enumerate}

\item $L(G) = \{w\in \Sigma^{\star} | S \productionstar w\}$
\item Notation:

\begin{enumerate}
\item Variables: A,B,C\ldots{}
\item Terminal: a,b,c,\ldots{}0,1,$\epsilon$
\item $U \productionstar V$ is defined as $\exists$ sequence
         $U_1..U_k$ , s.t. $U \Rightarrow U_1 \Rightarrow  U_2
         \Rightarrow ... \Rightarrow U_k \rightarrow V$
\end{enumerate}

\end{enumerate}
\subsection{Chomsky Normal form (CNF)}
\label{sec-3.2}

\begin{itemize}
\item All rules have the form

\begin{itemize}
\item $A\rightarrow BC$, where $B,C$ cannot be $S$
\item $A\rightarrow a$
\item if $A\rightarrow \epsilon$ then $A = S$
\end{itemize}

\item Lemma: Any CFG can be written in CNF
\end{itemize}
\subsection{Deterministic Push Down Automotas}
\label{sec-3.3}

\begin{itemize}
\item $M = (Q,\Sigma, \Gamma, \delta, q_0, F)$

\begin{itemize}
\item $\delta: Q \times \Sigma_\epsilon \times \Gamma_\epsilon
       \rightarrow Q \times \Gamma_\epsilon$
\end{itemize}

\end{itemize}
\subsection{Non-Deterministic Push Down Automotas (PDA)}
\label{sec-3.4}

\begin{itemize}
\item $M = (Q,\Sigma, \Gamma, \delta, q_0, F)$

\begin{itemize}
\item $\delta: Q \times \Sigma_\epsilon \times \Gamma_\epsilon
       \rightarrow Pow(Q \times \Gamma_\epsilon)$
\end{itemize}

\end{itemize}
\subsection{Relating PDA to CFL}
\label{sec-3.5}

\begin{itemize}
\item A language is context free iff $\exists$ a PDA that recognizes it

\begin{itemize}
\item Lemma: If A is CF, then $\exists$ a PDA,$M$, s.t. $A = L(M)$
\item Lemma: $\forall$ PDA, $M$, $\exists$ CFL, $G$, s.t. $L(G)=L(M)$

\begin{itemize}
\item Proof Idea: Make a conical PDA (while preserving acceptance)
         as follows

\begin{enumerate}
\item 1 accept states
\item Stack is empty when accepting
\item Every transition either push or pops but not both
\end{enumerate}

\end{itemize}

\end{itemize}

\end{itemize}
\subsection{Pumping Lemma for CFL's}
\label{sec-3.6}

   \underline{Theorem}: If A is a CFL, then $\exists p \geq 0, p\in\mathbb{Z}
   s.t. \forall s\in A : |S| \geq p \implies \exists$ a paritition
   $s=uvxyz$ that satisfy the follow conditions 

\begin{enumerate}
\item $\forall i \geq 0, uv^ixy^iz \in A$
\item $|vy| > 0$
\item $|vxy| \leq p$
\end{enumerate}
   \underline{Proof}:
\begin{enumerate}
\item Let G be a CFG s.t. $A=L(G)$
\item Let $b$ be the max length of the right side of a rule $\in R$
      and assume $b \geq 2$

\begin{enumerate}
\item If $b < 2$ the language must be finite thus the pumping lemma
         is trivially true
\end{enumerate}

\item Consider the derivation tree if the tree height is $h$, then the
      length of the generated string,s, $\leq b^h$
\item Let p = b$^{\mathrm{|V|+1}}$, where $V =$ set of variables
\item Observe that for any $s\in A$ that $|S| \geq p \implies h
      \geq |v|+1$
\item Choose the `smallest' derivation tree by height for s
\item The longest path has length $|v|+1$ and vists $|v|+2$ variables
\item Note that $|v|<$ variables visted, thus by the \textbf{Pidgen Hole       Principle} there must be at least 1 variable repeated
\item Thus There is a cycle in the production process strings which
      can then be repeated an indefiniate amount of times in the form
      $uv^ixy^iz \square$
\end{enumerate}
\subsection{Closure Properties of CFL's}
\label{sec-3.7}

\begin{enumerate}
\item $A_1 \cap A_2$
\item $A_1 \cdot A_2$
\item $A_1^{\star}$
\item Closure under substituion
\end{enumerate}
\subsubsection{Theorems for Closure}
\label{sec-3.7.1}

    Let $G_i = (V_i, \Sigma_i, R_i, S_i)$ for $i=1,2$ and $A_i = L(G_i)$ \\
    Without loss of generality, assume $V_1 \cap V_2 = \emptyset \wedge
    S_3$
    
\begin{itemize}

\item Theorem: If A$_1$ and A$_2$ are CFL's, then A$_1$ $\cup$ A$_2$ is a CFL\\
\label{sec-3.7.1.1}

Proof:
\begin{enumerate}
\item Let $G_i = (V_i, \Sigma_i, R_i, S_i)$ for $i=1,2$ and $A_i = L(G_i)$
\item Without loss of generality, assume $V_1 \cap V_2 = \emptyset \wedge S_3 \notin V_1 \cup V_2$
\item Construct $G_3 = (V_1 \cup V_2 \cup \{S_3\}, \Sigma_1 \cup
       \Sigma_2, R_3, S_3)$ with $R_3= R_1 \cup R_2 \cup \{S_3 \rightarrow S_1|S_2\}. \square$
\end{enumerate}

\item Theorem: If A$_1$ and A$_2$ are CFL's then A$_1$ \cdot A$_2$ is a CFL\\
\label{sec-3.7.1.2}

     Proof:     
\begin{enumerate}
\item $\notin V_1 \cup V_2$
\item Construct $G_3 = (V_1 \cup V_2 \cup \{S_3\}, \Sigma_1 \cup
        \Sigma_2, R_1 \cup R_2 \cup \{S_3 \rightarrow S_1 S_2\}, S_3)$
\end{enumerate}

\item Theorem: If A$_1$ and A$_2$ are CFL's then  A$_1$$^{\star}$ is a CFL\\
\label{sec-3.7.1.3}

     Proof:
     Construct $G_3 = (V_1 \cup \{S_3\}, \Sigma_1, R_1 \cup \{S_2 \rightarrow S_1 S_2 | \epsilon \})$
\end{itemize} % ends low level
\subsection{CYK algorithm}
\label{sec-3.8}

\subsubsection{Dynamic Programming (sequential optimization)}
\label{sec-3.8.1}

\begin{itemize}
\item Richard Bellman 1950
\item sequential decision making
\item extensive form games
\item optimal control theory
\item Dijkstra's Algorithm
\end{itemize}
\subsubsection{Algorithm}
\label{sec-3.8.2}

    Is $G \productionstar$ w true or false?
   
    $G = (V, \Sigma, R, S)$,
    Put G into Chomsky Normal Form,
    $w \in \Sigma^{ \star }$

    Cocke, Schwartz, Younger, Kasame
  

\begin{verbatim}
"""
Preconditions:
let the input be a string S consisting of n characters: a1 ... an.
let the grammar contain r nonterminal symbols R1 ... Rr.
This grammar contains the subset Rs which is the set of start symbols.
let P[n,n,r] be an array of booleans. Initialize all elements of P to false.
"""

for each i = 1 to n:
    for each unit production Rj -> ai:
        P[i][1][j] = true
for each i = 2 to n: # Length of span
    for j in range(1,n-i+2): # Start of span
        for k in range(1,i): # Partition of span
            A = filter(RA -> RB RC, G) # A = list of productions s.t. RA -> RB RC
            for production in A:
                if P[j][k][B] and P[j+k][i-k][C]:
                    P[j][i][A] = True
if any P[1][n]: #x is iterated over the set s, where s are all the indices for Rs)
    print 'S is member of language'
else:
    print 'S is not member of language'
\end{verbatim}


\section{Context Sensitive Languages}
\label{sec-4}

\subsection{Formal Definition:  Non-Contracting Grammars}
\label{sec-4.1}

   $G=(V,\Sigma,R,S)$
\begin{enumerate}
\item V is finite set of \underline{variables}
\item $\Sigma$  is a finite set of \underline{terminals} and $\Sigma\capV=
      \emptyset$
\item $\alpha\rightarrow\beta$
\item $|\alpha|\leq|\beta|$
\end{enumerate}
\subsection{Normal Form Definition}
\label{sec-4.2}

   3) R is a finite set of rules of the form $\alpha A\beta
      \rightarrow\alpha\gamma\beta$ in which $A$ is a variable and
      $\alpha,\beta,\gamma$ are strings of terminals and variables.

\begin{enumerate}
\item $\alpha,\beta\in(V\cup\Sigma)^{\star}$
\item $\gamma\in(V\cup\Sigma)^{\star}-{\epsilon}$
\item $S\in V$ is the start \underline{variable}
\item One additional rule allowed $S\rightarrow \epsilon$ and $S$ is not on the right side of any rule
\end{enumerate}
\subsection{Linear Bounded Automaton}
\label{sec-4.3}

\subsubsection{Informal Definition}
\label{sec-4.3.1}

    Has no stack but can read/write anywhere on the input string
\section{Turing Machines}
\label{sec-5}

\subsection{Formal Def}
\label{sec-5.1}

   A \underline{Turing Machine} is a 7-tuple $(Q,\Sigma,\Gamma,\delta,q_0,
   q_{accept},q _{reject})$ with $Q\Sigma,\Gamma\equiv$ non empty sets

\begin{itemize}
\item Q is set of \underline{states}
\item $\Sigma$ is the \underline{input alphabet}, which does not contain the blank
     symbol \_
\item $\Gamma$ is the \underline{tape alphabet}, in which $\_\in\Gamma$ and $\Sigma
     \subset\Gamma$
\item $\delta$ : $Q\times\Gamma\rightarrow Q\times\Gamma\times
     \{L,R\}$ is the \underline{transition function}
\item $q_0\in Q$ is the \underline{start state}
\item $q_{accept}\in Q$ is the \underline{accept state}
\item $q_{reject}\in Q$ is the \underline{reject state}
\item $q_{accept}\neq q_{reject}$
\end{itemize}
\subsection{Configurations}
\label{sec-5.2}

   A configuration of the turing machine $\equiv c_i =(q_i,p_i,t_i)$,
   where $q_i \in Q,p_i$ is the head pos, and $t_i\in \Gamma^{\star}$
   is the tape contents.

   Notice that configurations are unique, and from them given the next
   input symbol one can determine the next configuration. i.e.
   \begin{equation}
   \delta(q_i,\gamma_i):(c_i,\gamma_i) \mapsto
   (q_{i+1},p_{i+1},D_{i+1}) \mapsto c_{i+1}
   \end{equation}
   for $D_{i+1}\in \{L,R\}$
   
\subsubsection{Halting Configs}
\label{sec-5.2.1}

    If either of the following type of configs is encountered, the turning
    machine halts and returns `accept' or `reject' respectively.
\begin{itemize}

\item Accept Config\\
\label{sec-5.2.1.1}

     \begin{equation}
     c_{accept} \iff q_i = q_{accept}
     \end{equation}

\item Reject Config\\
\label{sec-5.2.1.2}

     \begin{equation}
     c_{reject} \iff q_i = q_{reject}
     \end{equation}

\end{itemize} % ends low level
\subsection{Turning Recognizable}
\label{sec-5.3}

   M is \underline{Recognizable} iff $\forall w \in L, M$
   accepts
\subsection{Turing Decidable}
\label{sec-5.4}

   M, is \underline{Decidable} iff $\forall w \in L, M$ accepts \underline{and}
   $\forall w \notin L, M$ rejects
\subsubsection{Co-Recognizablity}
\label{sec-5.4.1}

\begin{enumerate}
\item Define: $Sigma^{\star} - A = \bar{A}$ is recognizable
\end{enumerate}
\begin{itemize}

\item Decidablility Theorem:\\
\label{sec-5.4.1.1}

\begin{enumerate}
\item A language is only Turing Decidable iff it is both recognizable
        and co-recognizable
\item If a language is not decidable then its compliment is not recognizable
\end{enumerate}
\end{itemize} % ends low level
\subsection{Turing Machine Variants:}
\label{sec-5.5}

   Note that none of these add any power
   
\begin{itemize}
\item Multi-Tape: $\delta:Q\times\Gamma^k\rightarrow Q\times\Gamma^k\times\{L,R\}^k$

\begin{itemize}
\item Emulate on single tape by striping and recording/marking virtual head position
\end{itemize}

\item Adding Stay: $\delta:Q\times\Gamma\rightarrow Q\times\Gamma\times\{L,R,S\}$ where S doesn't move the head

\begin{itemize}
\item Emulate by moving left and then right while not changing tape contents
\end{itemize}

\item Non-Deterministic:
\end{itemize}
\subsection{Universial Turing Machine}
\label{sec-5.6}

   Turing machine that take a turing machine, M, as a string encoding
   denoted as $\langle M \rangle$

\subsection{Undecidability}
\label{sec-5.7}

\subsubsection{Halting Problem}
\label{sec-5.7.1}

\begin{enumerate}
\item Theorem: $A_{TM}$ is not Turing Decidable
\item Proof by Contradiction:

\begin{enumerate}
\item Suppose $A_{TM}$ were decidable.
\item Let H be a TM that decides $A_{TM}$

\begin{enumerate}
\item $H(\langle M, w \rangle) =$

\begin{enumerate}
\item accept if $w \in L(M)$
\item reject if $w \notin L(M)$
\end{enumerate}

\end{enumerate}

\item Construct a TM, D, which uses H and give the opposite result

\begin{enumerate}
\item $D \equiv$ on input $\langle M \rangle$, in which M is a
             T.M

\begin{enumerate}
\item Run H on input $\langle M, \langle M \rangle \rangle$
\item Return the opposite of what H outputs
\end{enumerate}

\end{enumerate}

\item Therefore  $D(\langle M \rangle):$

\begin{enumerate}
\item accept if M rejects $\langle M\rangle$
\item reject if M accepts $\langle M\rangle$
\end{enumerate}

\item Note that this implies $D(\langle D \rangle):$

\begin{enumerate}
\item accept if D rejects $\langle D\rangle$
\item reject if D accepts $\langle D\rangle$
\end{enumerate}

\item Notice that this is a contradiction, thus H and D cannot exist
\end{enumerate}

\end{enumerate}
\subsection{Reduction}
\label{sec-5.8}

   Using a language that is known to be undecidable prove that another
   language is not undecidable.

\subsubsection{Map Reductions}
\label{sec-5.8.1}

\begin{itemize}

\item Computable Functions\\
\label{sec-5.8.1.1}

     A function $f:\Sigma^{\star} \rightarrow \Sigma^{\star}$ is
     called \underline{computable} iff $\exists$ a TM, M, s.t. M halts $\forall
     w\in \Sigma^{\star}$, and after halting, $f(w)$ appears alone on
     the tape.

\item Mapping reducible\\
\label{sec-5.8.1.2}

     A language A is called \underline{mapping reducible} to language B, written
     $A \leq_M B$ iff: 
     \begin{equation}
     \exists (f: \Sigma^{\star} \rightarrow
     \Sigma^{\star})
     , \forall w \in \Sigma^{\star} : [w \in A \iff
     f(w) \in B ]
     \end{equation}
\begin{itemize}

\item Theorems\\
\label{sec-5.8.1.2.1}

\begin{enumerate}
\item Note the rule of thumb for these theorems is that $\leq_M$
         more or less reflects the Chomsky Heirarchy:
\item \textbf{Theorem}: $[A\leq_M B] \implies$ [If B is Turing Decidable, then A is
         turing Decidable]
\item \textbf{Collary}: $[A\leq_M B] \implies$ [If A is Turing Undecidable, then B
         is Turing Undecidable]
\item \textbf{Theorem}: $[A\leq_M B] \implies$ [If B is Turing
         Recognizable then A is Turing Recognizable]
\item \textbf{Collary}: $[A\leq_M B] \implies$ [If A is Turing
         Unrecognizable then B is Turing Unrecognizable]
\end{enumerate}
\end{itemize} % ends low level
\end{itemize} % ends low level
\subsubsection{Rice's Theorem:}
\label{sec-5.8.2}

\begin{enumerate}
\item If P is a set of TM's with a property that satisfies:

\begin{enumerate}
\item $\forall$ TM's $M$ and $M_2$ s.t. $L(M_1) = L(M_2)$,
          $\langle M_1 \rangle \in P$ iff $\langle M_2 \rangle \in P$
\item $\exists$ TM M$_1$ and M$_2$ for which $\langle M_1 \rangle \in
          P$ and $\langle M_2 \rangle \notin P$
\end{enumerate}

\item Then the language of P is \underline{not} Turing Decidable
\end{enumerate}
\section{Kolmogorov Complexity}
\label{sec-6}

\begin{itemize}
\item How `Small'(state number) of a turing machine is needed to
    generate a given string
\end{itemize}
\section{Complexity Theory:}
\label{sec-7}

\subsection{Definition:}
\label{sec-7.1}

\begin{enumerate}
\item \textbf{Time Complexity}: How many steps does a Turing machine take to decide?
\item \textbf{Space Complexity}: How much space does a turing machine need on the tape to decide?
\item Let $f(n) \equiv$ the max number of steps for a TM to decide w,
      with $n=|w|$

\begin{enumerate}
\item $f(n) = O(g(n)) \iff \exists (c,n_0) \in \mathbb{R} \times \mathbb{Z}$ s.t. $\forall n \geq
         f(n), f(n) \leq c g(n)$
\item Which is equivelent to: 
         \begin{equation}
         \forall w \in \Sigma^{\star}: \abs{w} \geq n_0 \iff f(\abs{w}) \leq c g(\abs{w})
         \end{equation}
\end{enumerate}

\end{enumerate}
\subsection{Complexity Classes}
\label{sec-7.2}

\begin{enumerate}
\item Let $t: \mathbb{N} \rightarrow \mathbb{N}$ be a function
\item Let $D \equiv$ set of all decidable TMs
\item Let $TIME[t(n)] \equiv {A \in D | \exists}$ a TM that decides A
      in $O(t(n))$ steps
\end{enumerate}
\subsubsection{P vs NP}
\label{sec-7.2.1}

\begin{itemize}

\item P\\
\label{sec-7.2.1.1}

     \begin{equation}
     P \equiv \cup_{k\in \mathbb{N}} TIME(n^k)
     \end{equation}
\begin{enumerate}
\item Notice that $TIME(n^k) \subset TIME(n^{k+1})$
\item Thus $P =$ all language for which $\exists$ a TM that decides
        in polynomial time
\end{enumerate}

\item NP\\
\label{sec-7.2.1.2}

     \begin{equation}
     NP \equiv \cup_{k\in \mathbb{N}} NTIME(n^k)
     \end{equation}
\begin{enumerate}
\item $NTIME(t(n)) \equiv \{A\in D | \exists$ a nondeterministic TM
        that decides A in $O(t(n))$ steps\}
\item Note that this appears to be equivlent to verification via bruteforce

\begin{enumerate}
\item $\textbf{Satisfyability Problem}$:
           A language $A\in NP \iff A$ is polinomially \underline{verifiable}.

\begin{enumerate}
\item $A$ is \underline{verifiable} iff $\exists$ a poly-time DTM that
              takes input $w$ for $A$ and the \underline{certificate} $c$ and
              decides if $w \in A$
\end{enumerate}

\end{enumerate}

\end{enumerate}

\item Million Dollar Question\\
\label{sec-7.2.1.3}

\begin{enumerate}
\item Does $P=NP$ ?
\end{enumerate}

\item Hardest NP problems\\
\label{sec-7.2.1.4}

\begin{enumerate}
\item A Problem, P, is \underline{NP-Complete} iff:

\begin{enumerate}
\item $P\in NP$
\item $\forall A \in NP, A \leq_P P$

\begin{enumerate}
\item Where $\leq_P$ is a mapping reduction where $f$ is
              computable in polynomial time
\end{enumerate}

\end{enumerate}

\end{enumerate}
\end{itemize} % ends low level

\end{document}