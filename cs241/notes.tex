% Created 2012-03-04 Sun 00:38
\documentclass[11pt]{article}
\usepackage[utf8]{inputenc}
\usepackage[T1]{fontenc}
\usepackage{graphicx}
\usepackage{longtable}
\usepackage{float}
\usepackage{wrapfig}
\usepackage{soul}
\usepackage{amssymb}
\usepackage{hyperref}


\title{CS 241 Midterm study guide}
\author{}
\date{04 March 2012}

\begin{document}

\maketitle

\setcounter{tocdepth}{3}
\tableofcontents
\vspace*{1cm}
\section{POSIX}
\label{sec-1}

\subsection{Examples}
\label{sec-1.1}

\section{System Calls}
\label{sec-2}

\section{Libraries}
\label{sec-3}

\subsection{strlen vs sizeof}
\label{sec-3.1}

\section{Pointers}
\label{sec-4}

\subsection{Pointer Operations}
\label{sec-4.1}

\subsubsection{Pointer Arithmetic}
\label{sec-4.1.1}

\subsection{NULL}
\label{sec-4.2}

\subsection{Function Pointers}
\label{sec-4.3}

\subsection{Strings}
\label{sec-4.4}

\subsubsection{What's the difference between char $c[ 80 ]$ and char* c}
\label{sec-4.4.1}

\begin{itemize}

\item What about when they're used in sizeof()?\\
\label{sec-4.4.1.1}

\end{itemize} % ends low level
\subsubsection{What's the difference between a string and a string literal?}
\label{sec-4.4.2}

\subsubsection{How do strcpy, strcat, and strncat work?}
\label{sec-4.4.3}

\section{Memory}
\label{sec-5}

\subsection{Memory Fragmentation}
\label{sec-5.1}

\subsubsection{Internal}
\label{sec-5.1.1}

\subsubsection{External}
\label{sec-5.1.2}

\subsection{Cache/Page Table replacment policies}
\label{sec-5.2}

\subsubsection{OPT}
\label{sec-5.2.1}

\subsubsection{LRU}
\label{sec-5.2.2}

\subsubsection{Working Set}
\label{sec-5.2.3}

\begin{itemize}

\item Locality\\
\label{sec-5.2.3.1}

\end{itemize} % ends low level
\subsubsection{Thrashing}
\label{sec-5.2.4}

\subsubsection{Belady's anomaly}
\label{sec-5.2.5}

\subsubsection{When is a process swapped out to disk}
\label{sec-5.2.6}

\subsection{Stack}
\label{sec-5.3}

\subsubsection{When is the stack full?}
\label{sec-5.3.1}

\subsection{Heap}
\label{sec-5.4}

\subsubsection{Page faults}
\label{sec-5.4.1}

\begin{itemize}

\item SEGFAULT\\
\label{sec-5.4.1.1}

\end{itemize} % ends low level
\subsubsection{How does malloc and free work?}
\label{sec-5.4.2}

\begin{itemize}

\item Memory Selection Algorithms\\
\label{sec-5.4.2.1}

\begin{itemize}

\item Implicit Free Lists\\
\label{sec-5.4.2.1.1}


\item Explicit Free Lists\\
\label{sec-5.4.2.1.2}


\item Segregated Free Lists\\
\label{sec-5.4.2.1.3}


\item Buddy System\\
\label{sec-5.4.2.1.4}

\end{itemize} % ends low level
\end{itemize} % ends low level
\subsection{Virtual Memory and Paging/Segmentation}
\label{sec-5.5}

\subsubsection{Virtual vs Physical memory}
\label{sec-5.5.1}

\begin{itemize}

\item Advantages of virtual memory\\
\label{sec-5.5.1.1}

\end{itemize} % ends low level
\subsubsection{Paging vs Segmentation}
\label{sec-5.5.2}

\subsubsection{Virtual to Physical address translation in multi-level page tables}
\label{sec-5.5.3}

\begin{itemize}

\item MMU\\
\label{sec-5.5.3.1}

\begin{itemize}

\item Example: How does the virtual memory subsystem know the exact location where a particular page is stored on disk, if it is swapped out of memory?\\
\label{sec-5.5.3.1.1}

\end{itemize} % ends low level

\item TLB\\
\label{sec-5.5.3.2}


\item Algorithm for address translation goes here\\
\label{sec-5.5.3.3}

\begin{itemize}

\item Example: Assuming a 32-bit address space and 4 KB pages, what is the virtual page \# and offset for virtual address 0xd34f6a5?\\
\label{sec-5.5.3.3.1}


\item Example: Suppose we have a 64-bit address space and 16 KB pages. How big is the page table of a single process? What if it was multi-level?\\
\label{sec-5.5.4.1}

\end{itemize} % ends low level
\end{itemize} % ends low level
\subsubsection{Advantages of multi-level page tables}
\label{sec-5.5.4}

\subsubsection{Determining optimal page size}
\label{sec-5.5.5}

\subsubsection{Calculating the number of pages per page table}
\label{sec-5.5.6}

\section{Threads and Processes}
\label{sec-6}

\subsection{Process}
\label{sec-6.1}

\subsubsection{Creating a process using fork()}
\label{sec-6.1.1}

\begin{itemize}
\item Starts new process with an incremented PC count
\end{itemize}
\subsubsection{exec()}
\label{sec-6.1.2}

\begin{itemize}

\item Example: Explain how a shell process can execute a different program.\\
\label{sec-6.1.2.1}

\end{itemize} % ends low level
\subsubsection{Orphans and Zombies}
\label{sec-6.1.3}

\subsection{Threads}
\label{sec-6.2}

\subsubsection{Shared Resources}
\label{sec-6.2.1}

\subsubsection{Creating a thread using pthread$_{\mathrm{create}}$()}
\label{sec-6.2.2}

\subsubsection{pthread$_{\mathrm{detach}}$() and pthread$_{\mathrm{join}}$}
\label{sec-6.2.3}

\begin{itemize}

\item Example: Explain how one process can wait on the return value of another process.\\
\label{sec-6.2.3.1}

\end{itemize} % ends low level
\subsubsection{Exiting a thread with out a thread library exit call}
\label{sec-6.2.4}

\begin{itemize}

\item How it happens: calling exit(), return, or termination\\
\label{sec-6.2.4.1}


\item Problems\\
\label{sec-6.2.4.2}

\end{itemize} % ends low level
\subsubsection{What are the maximum number of threads that can be run concurrently? How is this number determined?}
\label{sec-6.2.5}

\subsection{Context Switching}
\label{sec-6.3}

\subsubsection{In Processes}
\label{sec-6.3.1}

\subsubsection{In Threads}
\label{sec-6.3.2}

\subsubsection{Kernel-Space vs User-Space thread managment}
\label{sec-6.3.3}

\subsection{Memory Consistency}
\label{sec-6.4}

\subsubsection{Shared memory}
\label{sec-6.4.1}

\begin{itemize}

\item Example: X is a global variable and initially X=0. What are the possible values for X after two threads both try to increment X?\\
\label{sec-6.4.1.1}

\end{itemize} % ends low level
\subsubsection{Locking, Blocking, and Semaphores}
\label{sec-6.4.2}

\begin{itemize}

\item Mutual exclusion\\
\label{sec-6.4.2.1}


\item Semaphore and mutex\\
\label{sec-6.4.2.2}


\item Designing a a lock system for concurrent programming\\
\label{sec-6.4.2.3}

\end{itemize} % ends low level
\subsubsection{POSIX wait()}
\label{sec-6.4.3}

\section{Scheduling}
\label{sec-7}

\subsection{Five state model: started ,running, ready, blocked, terminated}
\label{sec-7.1}

\subsection{Scheduling schemes}
\label{sec-7.2}

\subsubsection{Wait Time}
\label{sec-7.2.1}

\subsubsection{Turnaround time}
\label{sec-7.2.2}

\subsubsection{Response time}
\label{sec-7.2.3}

\subsubsection{Preempting}
\label{sec-7.2.4}

\subsubsection{Quanta}
\label{sec-7.2.5}

\subsubsection{Fairness, progress guarentees, and interactive systems}
\label{sec-7.2.6}

\subsubsection{Schemes}
\label{sec-7.2.7}

\begin{itemize}

\item Round Robin\\
\label{sec-7.2.7.1}

\begin{itemize}

\item Quanta length vs performance\\
\label{sec-7.2.7.1.1}

\end{itemize} % ends low level

\item First Come First Serve (FCFS)\\
\label{sec-7.2.7.2}


\item Pre-emptive SJF\\
\label{sec-7.2.7.3}


\item Non-preemptive\\
\label{sec-7.2.7.4}

\begin{itemize}

\item Smallest Initial response time?\\
\label{sec-7.2.7.4.1}


\item Smallest Initial wait?\\
\label{sec-7.2.7.4.2}


\item Smallest Initial turnaround time?\\
\label{sec-7.2.7.4.3}


\item smalled average wait time?\\
\label{sec-7.2.7.4.4}


\item longest average wait time?\\
\label{sec-7.2.7.4.5}

\end{itemize} % ends low level
\end{itemize} % ends low level
\subsection{Execution Order}
\label{sec-7.3}

\subsection{Starvation}
\label{sec-7.4}

\subsection{Blocking}
\label{sec-7.5}

\subsection{Signals and Interrupts}
\label{sec-7.6}

\subsubsection{Explain how re-entrant functions are used in C.}
\label{sec-7.6.1}

\subsection{Convoy Effect}
\label{sec-7.7}


\end{document}