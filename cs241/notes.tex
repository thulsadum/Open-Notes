% Created 2012-05-10 Thu 18:04
\documentclass[11pt]{article}
\usepackage[utf8]{inputenc}
\usepackage[T1]{fontenc}
\usepackage{graphicx}
\usepackage{longtable}
\usepackage{float}
\usepackage{wrapfig}
\usepackage{soul}
\usepackage{amssymb}
\usepackage{hyperref}


\title{CS 241 Midterm study guide}
\author{}
\date{10 May 2012}

\begin{document}

\maketitle

\setcounter{tocdepth}{3}
\tableofcontents
\vspace*{1cm}
\section{POSIX}
\label{sec-1}

\subsection{Examples}
\label{sec-1.1}

\section{System Calls}
\label{sec-2}

\section{Libraries}
\label{sec-3}

\subsection{strlen vs sizeof}
\label{sec-3.1}

\section{Pointers}
\label{sec-4}

\subsection{Pointer Operations}
\label{sec-4.1}

\subsubsection{Pointer Arithmetic}
\label{sec-4.1.1}

\subsection{NULL}
\label{sec-4.2}

\subsection{Function Pointers}
\label{sec-4.3}

\subsection{Strings}
\label{sec-4.4}

\subsubsection{What's the difference between char $c[ 80 ]$ and char* c}
\label{sec-4.4.1}

\begin{itemize}

\item What about when they're used in sizeof()?\\
\label{sec-4.4.1.1}

\end{itemize} % ends low level
\subsubsection{What's the difference between a string and a string literal?}
\label{sec-4.4.2}

\subsubsection{How do strcpy, strcat, and strncat work?}
\label{sec-4.4.3}

\section{Memory}
\label{sec-5}

\subsection{Memory Fragmentation}
\label{sec-5.1}

\subsubsection{Internal}
\label{sec-5.1.1}

    Interprocesses fragmentation causes by poor ``mallocing'' (doesn't
    have malloc)
\subsubsection{External}
\label{sec-5.1.2}

    Physical to Virtual fragmentation managed by the kernel
\subsection{Cache/Page Table replacment policies}
\label{sec-5.2}

\subsubsection{OPT}
\label{sec-5.2.1}

    Theortically optimal page replacement algorithm. Swaps out page
    used furtherst in the future\ldots{}.this is impossible for general systems
\subsubsection{LRU}
\label{sec-5.2.2}

    Least recently used\ldots{}.see cs232 notes
\subsubsection{Working Set}
\label{sec-5.2.3}

    The complete set of data needed to complete an operation. Often
    very localized in either time or space.
\begin{itemize}

\item Locality\\
\label{sec-5.2.3.1}

     see cs232
\end{itemize} % ends low level
\subsubsection{Thrashing}
\label{sec-5.2.4}

    Access pattern that assures high miss rate in the cache due to
    limitations of replacment algorithm
\subsubsection{Belady's anomaly}
\label{sec-5.2.5}

    Sometimes as the size of the cache increases, miss rate also
    increases. Proved for page faults using FIFO
\subsubsection{When is a process swapped out to disk}
\label{sec-5.2.6}

    When it is evicted by the replacment policy
    
\subsection{Stack}
\label{sec-5.3}

\subsubsection{When is the stack full?}
\label{sec-5.3.1}

    When it meets the heap in the address space
\subsection{Heap}
\label{sec-5.4}

   Dynamic memory
\subsubsection{Page faults}
\label{sec-5.4.1}

\begin{itemize}

\item SEGFAULT\\
\label{sec-5.4.1.1}

\end{itemize} % ends low level
\subsubsection{How does malloc and free work?}
\label{sec-5.4.2}

\begin{itemize}

\item Memory Selection Algorithms\\
\label{sec-5.4.2.1}

\begin{itemize}

\item Implicit Free Lists\\
\label{sec-5.4.2.1.1}


\item Explicit Free Lists\\
\label{sec-5.4.2.1.2}


\item Segregated Free Lists\\
\label{sec-5.4.2.1.3}


\item Buddy System\\
\label{sec-5.4.2.1.4}

\end{itemize} % ends low level
\end{itemize} % ends low level
\subsection{Virtual Memory and Paging/Segmentation}
\label{sec-5.5}

\subsubsection{Virtual vs Physical memory}
\label{sec-5.5.1}

\begin{itemize}

\item Advantages of virtual memory\\
\label{sec-5.5.1.1}

\end{itemize} % ends low level
\subsubsection{Paging vs Segmentation}
\label{sec-5.5.2}

\subsubsection{Virtual to Physical address translation in multi-level page tables}
\label{sec-5.5.3}

\begin{itemize}

\item MMU\\
\label{sec-5.5.3.1}

\begin{itemize}

\item Example: How does the virtual memory subsystem know the exact location where a particular page is stored on disk, if it is swapped out of memory?\\
\label{sec-5.5.3.1.1}

\end{itemize} % ends low level

\item TLB\\
\label{sec-5.5.3.2}


\item Algorithm for address translation goes here\\
\label{sec-5.5.3.3}

\begin{itemize}

\item Example: Assuming a 32-bit address space and 4 KB pages, what is the virtual page \# and offset for virtual address 0xd34f6a5?\\
\label{sec-5.5.3.3.1}


\item Example: Suppose we have a 64-bit address space and 16 KB pages. How big is the page table of a single process? What if it was multi-level?\\
\label{sec-5.5.4.1}

\end{itemize} % ends low level
\end{itemize} % ends low level
\subsubsection{Advantages of multi-level page tables}
\label{sec-5.5.4}

\subsubsection{Determining optimal page size}
\label{sec-5.5.5}

\subsubsection{Calculating the number of pages per page table}
\label{sec-5.5.6}

\section{Threads and Processes}
\label{sec-6}

\subsection{Process}
\label{sec-6.1}

\subsubsection{Creating a process using fork()}
\label{sec-6.1.1}

\begin{itemize}
\item Starts new process with an incremented PC count
\end{itemize}
\subsubsection{exec()}
\label{sec-6.1.2}

\begin{itemize}

\item Example: Explain how a shell process can execute a different program.\\
\label{sec-6.1.2.1}

\end{itemize} % ends low level
\subsubsection{Orphans and Zombies}
\label{sec-6.1.3}

\subsection{Threads}
\label{sec-6.2}

\subsubsection{Shared Resources}
\label{sec-6.2.1}

\subsubsection{Creating a thread using pthread$_{\mathrm{create}}$()}
\label{sec-6.2.2}

\subsubsection{pthread$_{\mathrm{detach}}$() and pthread$_{\mathrm{join}}$}
\label{sec-6.2.3}

\begin{itemize}

\item Example: Explain how one process can wait on the return value of another process.\\
\label{sec-6.2.3.1}

\end{itemize} % ends low level
\subsubsection{Exiting a thread with out a thread library exit call}
\label{sec-6.2.4}

\begin{itemize}

\item How it happens: calling exit(), return, or termination\\
\label{sec-6.2.4.1}


\item Problems\\
\label{sec-6.2.4.2}

\end{itemize} % ends low level
\subsubsection{What are the maximum number of threads that can be run concurrently? How is this number determined?}
\label{sec-6.2.5}

\subsection{Context Switching}
\label{sec-6.3}

\subsubsection{In Processes}
\label{sec-6.3.1}

\subsubsection{In Threads}
\label{sec-6.3.2}

\subsubsection{Kernel-Space vs User-Space thread managment}
\label{sec-6.3.3}

\subsection{Memory Consistency}
\label{sec-6.4}

\subsubsection{Shared memory}
\label{sec-6.4.1}

\begin{itemize}

\item Example: X is a global variable and initially X=0. What are the possible values for X after two threads both try to increment X?\\
\label{sec-6.4.1.1}

\end{itemize} % ends low level
\subsubsection{Locking, Blocking, and Semaphores}
\label{sec-6.4.2}

\begin{itemize}

\item Mutual exclusion\\
\label{sec-6.4.2.1}


\item Semaphore and mutex\\
\label{sec-6.4.2.2}


\item Designing a a lock system for concurrent programming\\
\label{sec-6.4.2.3}

\end{itemize} % ends low level
\subsubsection{POSIX wait()}
\label{sec-6.4.3}

\section{Scheduling}
\label{sec-7}

\subsection{Five state model: started ,running, ready, blocked, terminated}
\label{sec-7.1}

\subsection{Scheduling schemes}
\label{sec-7.2}

\subsubsection{Wait Time}
\label{sec-7.2.1}

\subsubsection{Turnaround time}
\label{sec-7.2.2}

\subsubsection{Response time}
\label{sec-7.2.3}

\subsubsection{Preempting}
\label{sec-7.2.4}

\subsubsection{Quanta}
\label{sec-7.2.5}

\subsubsection{Fairness, progress guarentees, and interactive systems}
\label{sec-7.2.6}

\subsubsection{Schemes}
\label{sec-7.2.7}

\begin{itemize}

\item Round Robin\\
\label{sec-7.2.7.1}

\begin{itemize}

\item Quanta length vs performance\\
\label{sec-7.2.7.1.1}

\end{itemize} % ends low level

\item First Come First Serve (FCFS)\\
\label{sec-7.2.7.2}


\item Pre-emptive SJF\\
\label{sec-7.2.7.3}


\item Non-preemptive\\
\label{sec-7.2.7.4}

\begin{itemize}

\item Smallest Initial response time?\\
\label{sec-7.2.7.4.1}


\item Smallest Initial wait?\\
\label{sec-7.2.7.4.2}


\item Smallest Initial turnaround time?\\
\label{sec-7.2.7.4.3}


\item smalled average wait time?\\
\label{sec-7.2.7.4.4}


\item longest average wait time?\\
\label{sec-7.2.7.4.5}

\end{itemize} % ends low level
\end{itemize} % ends low level
\subsection{Execution Order}
\label{sec-7.3}

\subsection{Starvation}
\label{sec-7.4}

\subsection{Blocking}
\label{sec-7.5}

\subsection{Signals and Interrupts}
\label{sec-7.6}

\subsubsection{Explain how re-entrant functions are used in C.}
\label{sec-7.6.1}

\subsection{Convoy Effect}
\label{sec-7.7}

   The slow down of traffic due to queuing slowing down the whole system


\hrule
Final
\hrule
\section{C Programming}
\label{sec-8}

\begin{enumerate}
\item What is POSIX?
\item What is a library function? What is a system call? What is the
    difference? Given an example of a pure library function and a pure system call.
\item How does pointer arithmetic work?
\item What is the * operator? What does it do? What is the \& operator?
    What does it do? What is a function pointer? How do you define a function pointer?
\item What functions have you learned about in CS 241 that take a function pointer as a parameter?
\item What is a “C string”? How is a “C string” represented in memory?
\item What is NULL?
\item What is the difference between strlen() and sizeof()?
\item What’s the difference between a stack and a heap variable? What
    about global and static variables?
\item How do malloc() and free() work?
\item What’s the difference between char c\footnote{FOOTNOTE DEFINITION NOT FOUND: 80 } and char *c? \ldots{}what
     about when they’re used in sizeof()?
\item What is the difference between a string and a string
     literal?
\item How do strcpy(), strcat(), strncpy(), and strncat() work?
\item How do printf() and scanf() work? What are the common
     formatting arguments?
\item How do you read a series of lines from a file or stdin using fgets()?
\end{enumerate}
\section{Memory}
\label{sec-9}

\begin{enumerate}
\item What is the difference between physical and virtual memory?
\item What are common memory allocation algorithms and what are the advantages of each?
\item How are virtual addresses translated to physical addresses in multi-level page tables?
\item How do page size and the number of levels of page tables affect
      the number of entries in a page table?
\item What is the difference between internal and external fragmentation?
\item What are the different page replacement policies and the
      advantages of each?
\item Describe how the buddy system works and the run time for its operations.
\item What is thrashing? When does it occur?
\item What causes a SEGFAULT and what happens when one occurs?
\item When is a process swapped out to disk?
\item What is the difference between the MMU and the TLB? Describe
       the function of each.
\item Name three benefits of virtual memory (as opposed to allowing
       programs to directly access physical memory).
\item Name one advantage of segmentation over paging, and one
       advantage of paging over segmentation.
\item How is a page table similar to an inode? What is the difference
       between these structures?
\item Assuming a 32-bit address space and 4 KB pages, what is the
       virtual page \# and offset for virtual address 0xd34f6a5?
\item Give an example of a page fault that is an error, and an
       example of a page fault that is not an error.
\item Assume LRU page eviction and three pages of physical
       memory. Describe what happens
\item when the application accesses virtual memory pages in this
       sequence: 3,4,5,4,1,6,9,3,9,8,4,8,8,2.
\item How many page faults occur in the above example?
\item Why are pages set to read-only in the copy-on-write technique?
\item Suppose we have a 64-bit address space and 16 KB pages. How big
       is the page table of a single process, if the system uses
       single-level page tables? What is the problem here? How would multi-level page tables help solve this problem?
\item Which page replacement scheme is better, OPT or LRU? Why?
\item Why does LRU not suffer from Belady’s anomaly?
\item How does the virtual memory subsystem know the exact location
       where a particular page is stored on disk, if it is swapped out of memory?
\item How is the working set computed? How is the notion of a working
       set useful for managing memory of processes?
\item Compare and contrast (give one benefit and one disadvantage)
       for: implicit, explicit, segregated, and buddy free lists.
\end{enumerate}
\section{Processes and Threads}
\label{sec-10}

\begin{enumerate}
\item What resources are shared between threads of the same process?
\item Invent some code using pthread$_{\mathrm{create}}$() statements. What could be
      its output?
\item What are the possible values for X after both threads complete
      execution? (X is a global
      variable and initially X = 0.)
\item What happens when a thread calls exit()?
\item What happens to a process’s resources when it terminates normally?
\item Describe what happens when a process calls fork(). Be able to trace
      through the code.
\item Under what conditions would a process exit normally?
\item Explain the actions needed to perform a process context switch.
\item Explain the actions needed to perform a thread process switch.
\item What are the advantages and disadvantages of kernel-level threads
       over user-level threads?
\item Compare the use of fork() to the use of pthread$_{\mathrm{create}}$().
\item In a multiprocessor system, what system characteristics will cause
       other threads of the same
       process to block?
\item How can a process become orphaned and what does the OS do with it?
       What's a zombie?
\item Write a piece of code using fork() to create a process tree of
       depth n, where each process (a node in the tree) except for the
       ``leaf'' processes has exactly m child processes.
\item Describe how to use the POSIX call wait().
\item Explain what happens when a process calls exec().
\item Explain how reentrant functions are used in C.
\item What are the maximum number of threads that can be run
       concurrently? How is this number determined?
\item If a process spawns a number of threads, in what order will these
       threads run?
\item Explain how to use pthread$_{\mathrm{detach}}$() and pthread$_{\mathrm{join}}$().
\item Explain how a shell process can execute a different program
       without using system().
\item Explain how one process can wait on the return value of another
       process.
\item Describe the transitions between running, ready and blocked in the
       5 state model.
\item Understand how pthread$_{\mathrm{exit}}$() differs from exit().
\end{enumerate}
\section{Scheduling}
\label{sec-11}

\begin{enumerate}
\item What is starvation? Which scheduling policies have the possibility
      of resulting in starvation?
\item Which scheduling algorithm results the smallest average wait time?
\item What scheduling algorithm has the longest average response time?
\item Define turnaround time, waiting time and response time in the
      context of scheduling algorithms.
\item What is the convoy effect?
\item Why do processes need to be scheduled?
\item How does bounded wait apply to scheduling?
\item Which scheduling algorithm minimizes average initial response time?
      Waiting time? Total response time?
\item Why is SJF/PSJF hard to implement in real systems?
\item What does it mean to preempt a process?
\item What does it mean for a scheduling algorithm to be preemptive?
\item Describe Round-Robin scheduling and its performance advantages
       and disadvantages.
\item Describe the First Come First Serve (FCFS) scheduling
       algorithm. Explain the performance advantages and disadvantages.
\item Describe the Pre-emptive and Non-preemptive SJF scheduling
       algorithms. Explain the performance advantages and disadvantages.
\item Describe the Preemptive Priority-based scheduling
       algorithm. Explain the performance advantages and disadvantages.
\item How does the length of the time quantum affect Round-Robin
       scheduling?
\item Define fairness in terms of scheduling algorithms. What are the
       fairness properties of each of the scheduling disciplines discussed in class?
\item Which scheduling algorithms guarantee progress?
\item A process was switched from running to ready state. Describe the
       characteristics of the scheduling algorithm being used.
\item Which properties of scheduling algorithms affect the performance
       of interactive systems?
\end{enumerate}
\section{Synchronization}
\label{sec-12}

\begin{enumerate}
\item What is the readers-writers problem?
\item What is the producers-consumers problem?
\item What is the dining philosopher problem?
\item Recognize a correct solution to the readers-writers problem, the
    producers-consumers problem, and the dining philosopher. Be able
    to identify and explain an error in a specific implementation of any of the classic synchronization problems.
\item What happens when readers are prioritized over writers in the
      classic “readers writer problem”? How about if writers are prioritized over readers?
\item What is required so that deadlock and starvation do not occur in
      the dining philosopher’s problem? Give examples of solutions.
\item What is the difference between starvation, deadlock, race
      conditions and critical sections? Describe each.
\item What would happen if a system’s hardware synchronization primitive were replaced with a software function?
\item Which type of variables must be protected against concurrent
    readers and writers in any combination?
\item Given two threads running example code that contains a critical
     section, be able to identify if progress and mutual exclusion are ensured.
\end{enumerate}
\section{Mutexes and Semaphores}
\label{sec-13}

\begin{enumerate}
\item Understand the common semaphore and mutex functions (sem$_{\mathrm{wait}}$(),
      sem$_{\mathrm{post}}$(), etc).
\item How does a semaphore perform, in both the parent and child, after
      a fork()?
\item What are proper and improper code replacements for a
     test$_{\mathrm{and}}$$_{\mathrm{set}}$() operation?
\item How does the internal counter of a POSIX semaphore work? What does
    it mean if the value of the semaphore is 1?
\item How can the reader-writer problem be solved using only POSIX mutexes?
\item Using only one mutex, is it possible to create a semaphore? If
      so, how? If not, why?
\item Understand how to solve the producer-consumer problem using
      mutexes and semaphores.
\item What is a buffer overflow? What is a buffer underflow? Understand how failures in synchronization could cause buffer over and underflows.
\item What is progress?
\item What is mutual exclusion?
\item What are condition variables? Understand how they can be used in
     code.
\item Understand how to fix deadlocks and starvation in code involving mutexes, semaphores, and conditional variables.
\end{enumerate}
\section{Processes and Deadlock}
\label{sec-14}

\begin{enumerate}
\item Define deadlock.
\item Define circular wait, mutual exclusion, hold and wait, and no
      preemption. How are these related to deadlock?
\item How would the implementation of a web server using threads differ from one using processes?
\item What can happen if synchronization in a multiple-threaded program is not programmed carefully?
\item Why might an operating system use a resource allocation graph?
\item What are the conditions of a deadlock? How could you guarantee that each one of these conditions can be prevented?
\item What does waitpid() do?
\item What are the approaches for solving deadlock?
\item What is the difference between Deadlock Prevention, Deadlock
      Detection \& Recovery, and Deadlock Avoidance? What deadlock handling mechanism would you use?
\item What are the components of a resource allocation graph?
\item What problem does the Banker’s Algorithm solve? Given a set of
     processes how would you use the Banker’s Algorithm?
\item What is a safe state and how can you determine if a system is in a
     safe state?
\end{enumerate}
\section{IPC}
\label{sec-15}

\begin{enumerate}
\item What is the difference between a FIFO and a pipe?
\item How would you redirect standard out to a file?
\item What is the difference between a pipe, a FIFO, and an ordinary file on disk?
\item What happens when two processes read and write to a memory
      mapped file?
\item Explain how two processes can share memory using shmem.
\item Explain how a process can set which signals are caught or
      ignored using a signal set.
\item How can one process send a signal to another?
\item Describe the purpose of a POSIX signal.
\item Some signals cannot caught or ignored. Which signals are they and
      why shouldn’t they be allowed to be caught?
\item What does “kill -<parameter> pid” do?
\item How is the function sigwait() used?
\item How does the function alarm() work?
\end{enumerate}
\section{Networking}
\label{sec-16}

\begin{enumerate}
\item When do you use the close() system call with sockets?
\item Discuss how a multithreaded web server running on a single
      processor system could be optimized using the process scheduling
      methods discussed in class. Which do you recommend?
\item How do select() and poll() work? What problem do they it solve?
\item Describe the Posix accept() function.
\item How does HTTP work?
\item Describe the services provided by TCP.
\item How does TCP connection establishment work?
\item Describe the services provided by UDP.
\item Explain the difference between a regular and a connected UDP socket.
\item How do sockets support the client-server model?
\item Which is better, UDP or TCP? Which one would you use?
\item How does the Domain Name System (DNS) work?
\item How does DNS use caching?
\item How is DNS related to IP?
\end{enumerate}
\section{File systems and I/O}
\label{sec-17}

\begin{enumerate}
\item Given a description of the block size and i-node structure, what is the maximum size of a file?
\item How many i-node operations are required to fetch a file at /path/to/file?
\item What information is stored in an i-node? What information isn’t?
\item What data structure best describes an i-node?
\item What are the advantages and disadvantages of an i-node based file system?
\item Given the description of an i-node file system, how many i-node
   accesses are required to read the entire contents of a file of a
   given size? How many blocks does this file consume on disk?
\item What is an advantage of a soft link over a hard link?
\item What is I/O polling? What are advantages and disadvantages?
\item Describe disk I/O access using DMA.
\item How are file descriptors shared between threads in a single
    process? How are they shared between after a process executes a fork()?
\item When the size of a block changes in an i-node based file system,
    how does this change the maximum size of a file?
\item How do polling and interrupt driven I/O differ? What are the
    advantages and disadvantages of each?
\item How does the page-out process work?
\item Understand how hard-links result in different file names affecting
    the same i-node.
\item If an i-node based file system has a certain number of direct and
    single-indirect blocks, how large is the file?
\item Where does fstat() look to find the information that it returns?
\item How does a file system use caching?
\end{enumerate}

\end{document}