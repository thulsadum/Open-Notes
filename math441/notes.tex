% Created 2012-04-08 Sun 02:55
\documentclass[11pt]{article}
\usepackage[utf8]{inputenc}
\usepackage[T1]{fontenc}
\usepackage{graphicx}
\usepackage{longtable}
\usepackage{float}
\usepackage{wrapfig}
\usepackage{soul}
\usepackage{amssymb}
\usepackage{hyperref}
\usepackage{header}

\title{Diff Eq Notes}
\author{}
\date{08 April 2012}

\begin{document}

\maketitle

\setcounter{tocdepth}{3}
\tableofcontents
\vspace*{1cm}


\section{Initial Definitions}
\label{sec-1}

\begin{itemize}
\item Definition:

\begin{itemize}
\item DE is an equation that describes the properties of an unkown
\end{itemize}

\item Ordinary DE:

\begin{itemize}
\item describes functions of 1 variable
\end{itemize}

\item Partial DE:

\begin{itemize}
\item describes multivariable functions
\end{itemize}

\item Notation:

\begin{itemize}
\item independent variable: y
\item dependent variable: t
\end{itemize}

\end{itemize}
\section{Operator Notation}
\label{sec-2}

  Definition:
  $\frac{d^n}{dt^n} = D^n \rightarrow f^{(n)} = D^n(f)$
\section{Linear Diff Equations}
\label{sec-3}

  Definition: For an operator,L, the DE: L(y) = 0 is linear iff:
\begin{itemize}
\item L(y$_1$+y$_1$) = L(y$_1$)+L(y$_2$)
\item L(cy) = cL(y)
\end{itemize}
\section{Stability}
\label{sec-4}

\subsection{Definition}
\label{sec-4.1}

\begin{enumerate}
\item Stability $\equiv$ a system in which long term behvaior does not
      depend within some variation on initial conditions
\item For Linear D.E. y is Stable iff:
    \begin{equation}
    \lim_{t\rightarrow \infty} y_h = 0 \pm \epsilon
    \end{equation}
\item For S.O.L.E

\begin{enumerate}
\item Stable iff either of the equivalent statments is true:

\begin{enumerate}
\item $Re(r) < 0$
\item $a,b,c >0$ or $a,b,c <0$
\end{enumerate}

\end{enumerate}

\end{enumerate}
\section{Initial Value Problems}
\label{sec-5}

  \begin{equation}
  IVP = \left \{
  \begin{array}{lr}
  DE\\
  y = C_1\\
  \vdots\\
  y^{(n)} = C_n
  \end{array}
  \right.
  \end{equation}
 
\section{Seperable DE}
\label{sec-6}

\subsection{Definition}
\label{sec-6.1}

\subsubsection{Form}
\label{sec-6.1.1}

    \begin{equation}
    \d{y}{t}=f(y)g(t)
    \end{equation}
\subsubsection{Solution}
\label{sec-6.1.2}

\begin{enumerate}
\item Let DE be $\d{y}{t}=f(y)g(t)$
\item Seperating terms: $\frac{dt}{f(y)}\d{y}{t}=\frac{dt}{f(y)}f(y)g(t) \implies \frac{dy}{f(y)}=g(t)dt$

\begin{enumerate}
\item If $f(y)^{-1}$ is not defined from some t, solve outside of
          that range and consider those t separately
\end{enumerate}

\item Integrate both sides: $\int \frac{dy}{f(y)}=\int g(t)dt$
\item If possible solve for $y(t)$
\end{enumerate}
 
\section{Homogenous Equations}
\label{sec-7}

\subsection{Form}
\label{sec-7.1}

   \begin{equation}
   \d{y}{t}=f(\frac{y}{t})
   \end{equation}
\subsection{Technique for Solving:}
\label{sec-7.2}

\begin{enumerate}
\item Let $v=\frac{y}{t} \implies y=vt$
\item Change of variable: $y'=v't+v \implies v't+v=f(v)$
\item Moving $v$ to left side: $t\d{v}{t}=f(v)-v = g(v)$
\item Seperating: $\frac{dv}{g(v)}=\frac{dt}{t}$
\item Solve using Seperable D.E. techniques
\end{enumerate}
\section{Exact Equations}
\label{sec-8}

\subsection{Form}
\label{sec-8.1}

   \begin{equation}
   \pd{}{t}\Psi(f(t),y(t)) = \pd{\Psi}{f}\d{f}{t} + \pd{\Psi}{y}\d{y}{t}
   \end{equation}
\subsection{Technique for Solving:}
\label{sec-8.2}

\begin{enumerate}
\item Suppose DE is of the form: $M(x,y) + N(x,y) y_x = 0$
\item If M$_y$ = N$_x$, then DE is an Exact Eq, solve for $\Psi(x,y) = \Psi( f(x), y(x))$
\end{enumerate}
\section{First Order Linear Eq}
\label{sec-9}

\subsection{Form}
\label{sec-9.1}

   \begin{equation}
   y' + p(t)y = f(t)
   \end{equation}
\subsection{Solving using Integration Factors}
\label{sec-9.2}

\begin{enumerate}
\item Let $\mu$ be a mult factor s.t. $\mu y' + \mu'y=g(t) \implies [\mu(t) y(t)]'= g(t)$
\item Thus $\mu'=\mu p(t) \implies \frac{d\mu}{\mu}=p(t)dt \implies
      \mu = e^{\int p(t)dt }$
\item Therfore
      \begin{equation}
      y(t)e^{\int p(t)dt}=\int g(t)dt
      \end{equation} $\square$
\end{enumerate}
\subsection{Bernoulli's equations}
\label{sec-9.3}

\subsubsection{Form}
\label{sec-9.3.1}

    \begin{equation}
    y'+p(t)y=q(t)y^n, n\in \mathbb{Z}
    \end{equation}
\subsubsection{Solution}
\label{sec-9.3.2}

\begin{enumerate}
\item Let $v=y^{2-n} \implies v'=(1-n)y^{-n}y'$
\item Thus $y'=\frac{v'}{1-n}$ and $y = y^n v$
\item Subsituting in Bernoulli equation: $\frac{v'}{1-n}y^n+p(t)y^nv=q(t)y^n$
\item Moving into standard form: 
       \begin{equation}
       v'+(1-n)p(t)v=(1-n)q(t)
       \end{equation}
\item Solve using Integration Factors $\square$
\end{enumerate}
\subsection{Picard Iteration}
\label{sec-9.4}

\subsubsection{Integral Equations}
\label{sec-9.4.1}

    Suppose $f$ is continous, then a function $y=\Phi(t)$
    solves the IVP iff $y=\Phi(t)$ solves the corresponding integral equation:
    \begin{equation}
    y(t)=y_0+\int_{t_0}^t f(s,y(s))ds
    \end{equation}
\subsubsection{Idea}
\label{sec-9.4.2}

\begin{enumerate}
\item Let $f(t) = \d{y}{t}$
\item Construct a sequence of functions $\{g_n(t) : n \geq 0, n\in \mathbb{Z}\}$ that converges to soln:

\begin{enumerate}
\item $y_0(t) = y_0$
\item $y_{n+1}(t)=y_0+ \int_{t_0}^tf(s,y_n(t))ds$
\end{enumerate}

\end{enumerate}
\subsection{Lipsichitz Condition}
\label{sec-9.5}

\begin{enumerate}
\item For $f(t,y) \in \mathbb{R}$, f is Lipsichitz iff $\exists L\in \mathbb{R}$:
      \begin{equation}
      \abs{f(t_1,y_1)-f(t_2,y_2)} \leq L\cdot \abs{(y_1-y_2)}
      \end{equation}
\item If $\Delta y \neq 0$ then this can be thought of as:
      \begin{equation}
      \abs{\frac{\Delta f}{\Delta y}} \leq L
      \end{equation}
\item \underline{Lemma}: if $f_y$ is bounded then f is Lipsichitz
\end{enumerate}
\subsection{Uniform Convergence (U.C.)}
\label{sec-9.6}

\subsubsection{Definition:}
\label{sec-9.6.1}

    A sequence of functions $\{f_n(t) : n\geq 0 ; n\in\mathbb{Z}\}$
    defined on the inverval $I$ \underline{uniformially converges} to $f(t)$ iff
    $\forall t > 0, \exists N\in \mathbb{Z}$ s.t. $\abs{f_n(t)-f(t)}
    < \epsilon$ everywhere on I $\forall n > N$
\subsubsection{}

    \underline{Theorem}: Given $\f_n(t)$ is continuous on I, if
    $\lim_{n\rightarrow \infty}{f_n(t)} \rightarrow f(t)$ with U.C, then:
\begin{enumerate}
\item $f$ is continuous
\item If $f_n$ is differtiable, then $f$ is differtiable and $f'_n$
          U.C. to $f'$
\item The limit is communitive with respect to integration
         \begin{equation}
         \lim_{n\rightarrow \infty}\int_I f_n(t)dt = \int_I \lim_{n\rightarrow \infty} f_n(t) dt
         \end{equation}
\end{enumerate}
\subsubsection{Weirstress M Test}
\label{sec-9.6.3}

    \underline{Theorem}: 
\begin{itemize}
\item If $\forall n \in I, \abs{f_n(t)} \leq M_n$ and if $\sum_{n=0}^\infty M_n< L$ for some $L\in \mathbb{R}$,
\item Then $\sum_{n=0}^\infty f_n(t)$ Converges Uniformially on I
\end{itemize}
\subsection{Existence Theorem}
\label{sec-9.7}

\begin{enumerate}
\item \underline{Claim}:

\begin{enumerate}
\item If:

\begin{enumerate}
\item $f(y)$ is continous
\item $f$ is Lipsichitz w.r.t. $y \in R\equiv \{\(t,y) :
            \abs{t-t_0}\leq T$ and $\abs{y-y_0} \leq k\}$
\item $\sum_{k=1}^\infty [y_k(t)-y_{k-1}(t)]$ converges uniformially
\end{enumerate}

\item Then: $\exists$ a solution to the IVP on the interval
         $\abs{t-t_0}\leq T_1=min(T,\frac{k}{m})$ where $\abs{f(t,y)} \leq M\in R$
\end{enumerate}

\item \underline{Proof}:

\begin{enumerate}
\item Converting the IVP to an I.E.: $y(t)=y_0+\int_{t_0}^t f(s,y(s))ds$
\item Note theat: $\abs{y_k(t)-y_{k-1}(t)} \leq \frac{M}{L}
         \frac{L^n (t-t_0)^n}{n!} \leq \frac{M}{L}\frac{L^n T_1^n}{n!}$
\item Define: $M_n\equiv\sum_{k=1}^\infty\frac{M}{L}\frac{(L T_1)^n}{n!} = \frac{M}{L}(e^{LT_1}-1)$
\item Apply the Weirstress M Test, because $\frac{M}{L}\frac{(L
         T_1)^n}{n!}$ converges, then $\sum_{k=1}^\infty[y_k-y_{k-1}]$ converges
\item Thus the series $\{y_n : n \geq 1\}$ converges uniformially
         on the interval.
\item Therefore $\exists$ a solution to the IVP $\square$
\end{enumerate}

\end{enumerate}
\subsection{Uniqueness Theorm}
\label{sec-9.8}

\begin{enumerate}
\item \underline{Claim}:

\begin{enumerate}
\item If $\Phi(t)$ and $\Psi(t)$ are solutions of $y'\equiv
         f(y,t) \in R$ and if $f$ is Lipseitz w.r.t. $y\in R$
\item Then $\abs{\Phi(t) - Psi(t)} \leq e^{L\abs{t-t_0}}
         \abs{\Phi(t_0) - \Psi(t_0)} = 0$

\begin{enumerate}
\item Because they solve the same I.V.P. $\abs{\Phi(t_0) -
            \Psi(t_0)} = 0$
\end{enumerate}

\item Equivalently: Then $\Psi(t) = \Psi(t)$
\end{enumerate}

\item \underline{Proof}:

\begin{enumerate}
\item $E\equiv \abs{\Phi(t)-\Psi(t)}^2$

\begin{enumerate}
\item Note that $E\geq 0$
\end{enumerate}

\item $\d{}{t}E=2(\Phi(t)-\Psi(t))(\Phi'(t)-\Psi'(t))$
\item $E'\stackrel{DE}{=}2(\Phi(t)-\Psi(t))(f(t,\Phi)-f(t,\Psi))$
\item $E'\stackrel{Lip}{\leq} 2\abs{\Phi(t)-\Psi(t)}L\abs{\Phi(t)-\Psi(t)}$
\item Thus $E' \leq 2LE \implies E'-2LE \leq 0\implies (E(t) e^{-2Lt})' \leq 0$

\begin{enumerate}
\item Note that E' is stricly decreasing
\end{enumerate}

\item Therefore: $e^{-t}E(t)\leq e^{2Lt_0} E(t_0) \implies
         E(t)\leq e^{2L(t-t_0)}E(t_0)$
\item Substituting: $\abs{\Phi(t)-\Psi(t)}^2 \leq e^{2L(t-t_0)} \abs{\Phi(t_0)-\Psi(t_0)}^2$
\item Because of absolute value: $\abs{\Phi(t)-\Psi(t)} \leq e^{2L(t-t_0)} \abs{\Phi(t_0)-\Psi(t_0)}$
\item Because they solve the same I.V.P. $\abs{\Phi(t_0) -
         \Psi(t_0)} = 0$
\item Thus $\Phi(t)=\Psi(t) \square$
\end{enumerate}

\end{enumerate}
\section{Autonomous Equations}
\label{sec-10}

\subsection{Form}
\label{sec-10.1}

   \begin{equation}
   y'=f(y)
   \end{equation}
\subsection{Anaylsis}
\label{sec-10.2}

\subsubsection{Differentiation Fields}
\label{sec-10.2.1}

\section{Second Order Linear Eq}
\label{sec-11}

\subsection{The Wronskian:}
\label{sec-11.1}

\begin{equation}
W(f,g)(t) = \left |
\begin{array}{cccc}
         f_1(x) & f_2(x) & \cdots & f_n(x) \\
         f_1'(x) & f_2'(x) & \cdots & f_n' (x)\\
         \vdots & \vdots & \ddots & \vdots \\
         f_1^{(n-1)}(x)& f_2^{(n-1)}(x) & \cdots & f_n^{(n-1)}(x)
\end{array} \right |,\qquad x\in I
\end{equation}

\subsection{Existence Theorem}
\label{sec-11.2}

\begin{enumerate}
\item Claim: For all D.E. there exists a $y(t)$ that satisfies it
      locally on some interval
\item Proof:

\begin{enumerate}
\item Let $y'=v \rightarrow v'=y''$        2) Therefore $v'=-py'-qy=-pv-qy$ , by plugging into the DE
\item In matrix form:
          \begin{equation}
          \left [
          \begin{array}{c}
          y\\v
          \end{array} \right ]'=
          \left [
          \begin{array}{lr}
          0 & 1\\
          -q & -p
          \end{array} \right ]
          \left [
          \begin{array}{c}
          y\\v
          \end{array} \right ]
          \end{equation}
\item Note that this is a linear first order matrix system which there
          is an existence therom for
\end{enumerate}

\end{enumerate}
\subsection{Uniqueness}
\label{sec-11.3}

\subsection{Second Order Linear Homogenous Diff Eq (S.O.L.H.D.E)}
\label{sec-11.4}

\subsubsection{Form}
\label{sec-11.4.1}

    \begin{equation}
    a(x)\dd{y}{t} + b(x)\d{y}{t} + c(x)y=0
    \end{equation}
\subsubsection{Theorm: The general solution to S.O.L.H.E}
\label{sec-11.4.2}

    Claim: The general soln of eq1$\equiv [y''+p(t)y'+g(t)y=0]$ is:
    \begin{equation}
    y_h=c_1y_1+c_2 y_2
    \end{equation}
\subsubsection{Proof:}
\label{sec-11.4.3}

\begin{itemize}

\item Q1:\\
\label{sec-11.4.3.1}

     Given y$_1$ and y$_2$ are solutions, why is $c_1y_1+c_2y_2$ a solution
\begin{enumerate}
\item $Eq1=D^2(y)+p(t)D(y)+q(t)y=0$
\item $Eq1=[D^2+p(t)D+q(t)]y=0$
\item Let $L=[D^2+p(t)D+q(t)]\rightarrow eq1\equiv L(y)=0$
\item Notice the L is a linear operator and thus obeys the
           superposition principle
\item Thus $y = c_1 y_1 +c_2 y_2$ is a solution $\square$
\end{enumerate}

\item Q2:\\
\label{sec-11.4.3.2}

     Given 2 indepent solutions y$_1$ and y$_2$ for the DE, $\forall$ IVP and its unique solution y, $\exists (c_1,c_2)
        \in \mathbb{C}^2$ s.t. $y=c_1y_1 + c_2y_2 \equiv \vec{y}\cdot\vec{c}$
        

\item Q3:\\
\label{sec-11.4.3.3}

\end{itemize} % ends low level
\subsubsection{Generating Second Solution}
\label{sec-11.4.4}

\begin{enumerate}
\item Claim: if $y_1\neq 0$ be a solution to the D.E. then,
       \begin{equation}
       y_2 = Cy_1\int \frac{e^{-\int p dt}}{y_1^2}
       \end{equation}
       and y$_2$ = solution independent of y$_1$
\item Proof:

\begin{enumerate}
\item Consider $(\frac{y_2}{y_1})' = \frac{y_1 y_2' - y_1'y_2}{y_1^2}=\frac{W(y_1,y_2)}{y_1^2}$
\item Given that $W'+p(t)W=0 \implies W(t)=ce^{-\int p(t) dt}$
\item Thus: $\int (\frac{y_2}{y_1})'dt = C \int \frac{e^{-\int p
          dt}}{y_1^2}$
\item Solving: $y_2=Cy_1\int \frac{e^{-\int p
          dt}}{y_1^2} \square$
\end{enumerate}

\end{enumerate}
\subsection{Second Order Linear Inhomogenious Diff Eq (S.O.L.I.D.E)}
\label{sec-11.5}

\subsubsection{Form}
\label{sec-11.5.1}

    \begin{equation}
    y^{(n)}(t)+p(t)y'(t)+q(t)y=f(t)
    \end{equation}
\subsubsection{General Solution}
\label{sec-11.5.2}

\begin{enumerate}
\item \underline{Claim}: The general soln of $y^{(n)}(t)+p(t)y'(t)+q(t)y=f(t)$ is:
        \begin{equation}
        y=y_h+k(t)
        \end{equation}

\begin{enumerate}
\item $y_h = c_1y_1+c_2y_2$ is the solution to the homogenous
           equation i.e. $f(t)=0$
\item Functional Offset $(k(t))$: variation or `offset' from the homogenous equation
\end{enumerate}

\item \underline{Proof}:

\begin{enumerate}
\item Sub-Claim: $y_h+k$ is a solution

\begin{enumerate}
\item Using Operator notation: $D^2y+pDy+qy=f \implies [D^2+pD+q](y)=f$
\item Let $L\equivD^2+pD+q \implies L(y)=f$

\begin{enumerate}
\item Note that L is linear
\end{enumerate}

\item $L(y_h+k) = L(y_h+k)=L(y_h)+L(y_p)$
\item $L(y_h)=0,L(y_p)=f \implies L(y_h+k)=f+0 = f \square$
\end{enumerate}

\item Sub-Claim: $\forall y_i$, if $y_i$ is a solution to the
           S.O.L.I.D.E, then $y_i=y_h+k$

\begin{enumerate}
\item $[L(y_i)=f$ and $L(k)=f]$ \$\implies L(y$_i$-k) = f-f=0\$
\item By Existence of S.O.L.H.D.E, $L(y_i-k)=0 \implies$ \$y$_i$-k=y$_h$\$
\item Thus $y_i=k+y_h \square$
\end{enumerate}

\end{enumerate}

\end{enumerate}
\subsubsection{Exponential Shift Law}
\label{sec-11.5.3}

    \begin{equation}
    P(D)[e^{\alpha u(t)}]=p(D+\alpha)u(t)[e^{\alpha t}]
    \end{equation}
\subsubsection{Expontial-Polynomial Functional offesets}
\label{sec-11.5.4}

\begin{itemize}

\item Form\\
\label{sec-11.5.4.1}

     \begin{equation}
     ay''+by'+cy=e^{\alpha t}g(t); \alpha\in\mathbb{C}
     \end{equation}

\item Characteristic Polynomial\\
\label{sec-11.5.4.2}

     \begin{equation}
     p(r)=aD^2+bD+c
     \end{equation}
     Note that the DE in Operator notation is: $[aD^2+bD+c]$

\item Finding Particular Solution for S.O.L.E\\
\label{sec-11.5.4.3}

     \underline{Theorem}:
\begin{itemize}
\item Let $k$ be s.t. $(r-\alpha)^k$ are roots of $p(\alpha)$
\item Then 
       \begin{equation}
       y_p=\frac{t^ke^{\alpha t}}{p^{(k)}(\alpha)}
       \end{equation}
\end{itemize}

\item Method of Undetermined Coefficents\\
\label{sec-11.5.4.4}

     \underline{Idea}: if f(t) is a comprised of strict multiplications (no division) sinusoidal,exponetials, and
     polynomials then the solution of the S.O.L.E with const
     coefficents is in terms of of the same types you began with.

     \underline{Cases}:
     
     if $f(t)=e^{\alpha t}$ (polynomial of $deg(k+m)$), then guess
     $y_p=e^{\alpha t}\Sigma_{j=0}^k C_jt^j$

\end{itemize} % ends low level
\subsubsection{Lagrange Variation of Parameters}
\label{sec-11.5.5}

\begin{itemize}

\item Equation:\\
\label{sec-11.5.5.1}

     \begin{equation}
     y_p=\int{\frac{y_1f(x)}{W(y_1,y_2)}dt}+\int{\frac{y_2f(x)}{W(y_1,y_2)}dt}
     \end{equation}

\item Derivation:\\
\label{sec-11.5.5.2}

     See General Derivation
\end{itemize} % ends low level
\subsection{Foulrier Transform}
\label{sec-11.6}

\subsection{Strum Comparison Theorem}
\label{sec-11.7}

   \underline{Theorem}: 
\begin{enumerate}
\item If:

\begin{enumerate}
\item $u''+q_1(t)u=0$ and $v''+q_2v =0$
\item $q_1>q_2$
\end{enumerate}

\item Then:

\begin{enumerate}
\item $u$ vanishes as some point between $2$ zeros of $v$
\end{enumerate}

\end{enumerate}
   
\section{General Linear Diff Eq and Variation of Parameters}
\label{sec-12}

\subsection{Form}
\label{sec-12.1}

   \begin{equation}
   y^{(n)}(x)+\sum_{k=0}^na_k(x)y^{(k)}(x)=f(x)
   \end{equation}
\subsection{Equation:}
\label{sec-12.2}

   \begin{equation}
   \sum_{k=0}^n[y_k(x)\int{\frac{W_k(x)}{W(X)}dx}]
   \end{equation}
   $W(x)\equiv$ Wronskian determinant of the fundamental system
   and $W_i(x)\equiv$ the Wronskian determinant of the fundamental system
   with the $i-th$ column replaced by $(0,0,\ldots,f(x))$   
\subsubsection{Derivation:}
\label{sec-12.2.1}


\subsection{Theorem for L.D.E}
\label{sec-12.3}

   If $u(t)+iv(t)$ is a solution to the D.E. then $u(t) \wedge v(t)$
   are solutions

\end{document}