\documentclass[12pt]{article}
\textwidth = 6.7 in
\textheight = 9.2 in
\oddsidemargin = 0.0 in
\evensidemargin = 0.0 in
\topmargin = 0.2 in
\headheight = 0.0 in
\headsep = 0.0 in
\parskip = 0.2in
\parindent = 0.0in

% ***********************************************************
% *********************** HEADER  ***************************
% ***********************************************************

\usepackage{amsmath} % AMS Math Package
\usepackage{amsthm} % Theorem Formatting
\usepackage{amssymb}	% Math symbols such as \mathbb
\usepackage{graphicx} % Allows for eps images
\usepackage{multicol} % Allows for multiple columns
\usepackage[dvips,letterpaper,margin=0.75in,bottom=0.5in]{geometry}
 % Sets margins and page size
\pagestyle{empty} % Removes page numbers
\makeatletter % Need for anything that contains an @ command 
\renewcommand{\maketitle} % Redefine maketitle to conserve space
{ \begingroup \vskip 10pt \begin{center} \large {\bf \@title}
	\vskip 10pt \large \@author \hskip 20pt \@date \end{center}
  \vskip 10pt \endgroup \setcounter{footnote}{0} }
\makeatother % End of region containing @ commands
\renewcommand{\labelenumi}{(\alph{enumi})} % Use letters for enumerate
% \DeclareMathOperator{\Sample}{Sample}
\let\vaccent=\v % rename builtin command \v{} to \vaccent{}
\renewcommand{\v}[1]{\ensuremath{\mathbf{#1}}} % for vectors
\newcommand{\gv}[1]{\ensuremath{\mbox{\boldmath$ #1 $}}} 
% for vectors of Greek letters
\newcommand{\uv}[1]{\ensuremath{\mathbf{\hat{#1}}}} % for unit vector
\newcommand{\abs}[1]{\left| #1 \right|} % for absolute value
\newcommand{\avg}[1]{\left< #1 \right>} % for average
\let\underdot=\d % rename builtin command \d{} to \underdot{}
\renewcommand{\d}[2]{\frac{d #1}{d #2}} % for derivatives
\newcommand{\dd}[2]{\frac{d^2 #1}{d #2^2}} % for double derivatives
\newcommand{\pd}[2]{\frac{\partial #1}{\partial #2}} 
% for partial derivatives
\newcommand{\pdd}[2]{\frac{\partial^2 #1}{\partial #2^2}} 
% for double partial derivatives
\newcommand{\pdc}[3]{\left( \frac{\partial #1}{\partial #2}
 \right)_{#3}} % for thermodynamic partial derivatives
\newcommand{\ket}[1]{\left| #1 \right>} % for Dirac bras
\newcommand{\bra}[1]{\left< #1 \right|} % for Dirac kets
\newcommand{\braket}[2]{\left< #1 \vphantom{#2} \right|
 \left. #2 \vphantom{#1} \right>} % for Dirac brackets
\newcommand{\matrixel}[3]{\left< #1 \vphantom{#2#3} \right|
 #2 \left| #3 \vphantom{#1#2} \right>} % for Dirac matrix elements
\newcommand{\grad}[1]{\gv{\nabla} #1} % for gradient
\let\divsymb=\div % rename builtin command \div to \divsymb
\renewcommand{\div}[1]{\gv{\nabla} \cdot #1} % for divergence
\newcommand{\curl}[1]{\gv{\nabla} \times #1} % for curl
\let\baraccent=\= % rename builtin command \= to \baraccent
\renewcommand{\=}[1]{\stackrel{#1}{=}} % for putting numbers above =
\newtheorem{prop}{Proposition}
\newtheorem{thm}{Theorem}[section]
\newtheorem{lem}[thm]{Lemma}
\theoremstyle{definition}
\newtheorem{dfn}{Definition}
\theoremstyle{remark}
\newtheorem*{rmk}{Remark}
% \def\rmdefault{bch} % Use Charter for main text font.

% ***********************************************************
% ********************** END HEADER *************************
% ***********************************************************
% =========================================================
\begin{document}

\begin{center}
  {\LARGE
    \textbf{Calc 3 Review}\\
    \normalsize
    \textnormal {$\tau = 2\pi$}
    \\[1ex]}

  
\end{center}

  % ------------------------------------------------------------
Part 1

Lines and Surfaces
\begin{align*}
  \textrm{Line: } \vec r(t) &= \vec r_0 + t*\vec v\\
  \textrm{Line: } \frac{x-x_0}{a} &= \frac{y-y_0}{b} = \frac{z-z_0}{c}\\
  \textrm{Line Segment: } \vec r(t) &= \vec (1-t)\vec r_0 + t*\vec
  r_1 \textrm{ for } 0 \leq t \leq 1 \\
  \textrm{Plane: } \vec n \cdot (\vec r - \vec r_0) &= 0 \\
  \textrm{Elliptic Paraboloid: } \frac{z}{c} &= \frac{x}{a}^2+\frac{y}{b}^2\\
  \textrm{Hyperbolic Paraboloid: } \frac{z}{c} &= \frac{x}{a}^2-\frac{y}{b}^2\\
  \textrm{Ellipsoid: }\frac{x}{a}^2+\frac{y}{b}^2+\frac{z}{c}^2&= 1\\
  \textrm{Cone: } \frac{x}{a}^2+\frac{y}{b}^2-\frac{z}{c}^2&= 0\\
  \textrm{Hyperboloid of 1 sheet: } \frac{x}{a}^2+\frac{y}{b}^2-\frac{z}{c}^2&= 1\\
  \textrm{Hyperboloid of 2 sheets: } \frac{x}{a}^2+\frac{y}{b}^2-\frac{z}{c}^2&= -1\\
\end{align*}

Vectors
\begin{align*}
|\vec r| &= \sqrt{a_x^2+a_y^2+a_z^2}\\
\vec{a} \cdot \vec{b} &= a_1*b_1 + b_2*b_3\\
\vec{a} \cdot \vec{b} &= |\vec a||\vec b|*\cos{\theta}, 0 \leq \theta \leq \tau \\
comp_a\vec b &=  \frac{\vec a \cdot \vec b}{|\vec a|} \\
proj_a \vec b &= comp_a \vec b * \frac{\vec a}{|\vec a|}\\
\vec a \times \vec b &= 
\begin{vmatrix}
  i   & j   & k \\
  a_1 & a_2 & a_3 \\
  b_1 & b_2 & b_3
\end{vmatrix}
\\
\textrm{Distance from point to a line} &= comp_{\vec n}\vec b \\
|\vec a \times \vec b | &= |\vec a| |\vec b| * \sin{\theta}\\
\end{align*}

Vector Functions
\begin{align*}
  \textrm{Arc Length} = \int_{a}^b|\vec {r(t)}| dt  
  \vec v &= \vec r’ , \vec a = \vec r’’ , v = |\vec v|\\
  \vec T &=\vec v/|\vec v|\\
  \vec B &= \vec v \times \vec a\\
  \vec B &= \vec T \times \vec N\\
  \vec N &= \vec N \times \vec B\\
  a_T &= comp_T \vec a = \vec a \cdot \vec T\\
  \vec w &= \vec a − a_T* \vec T \\
  \vec w &= a_N·\vec N\\
  a_N &= |\vec w|\\
  \vec N &=\frac{\vec w}{a_N} \\
  \kappa &= \frac{|\vec a \times \vec v |}{|\vec v|^3}
\end{align*}
% ================================================================
Part 2
% ================================================================
Multivariable Differentiation
\begin{align*}
  f_x(x,y) &= f_x = \frac{\partial f}{\partial x}\\ 
  f_{xy} &= f_{yx} \\
  dz &= \frac{\partial z}{\partial x}dx + \frac{\partial z}{\partial y}dy\\
  \textrm{Linear Approx:} f(x,y) &\approx f(a,b) + dz\\
  \textrm{Chain Rule:}
  \frac{\partial z}{\partial s} &=
  \frac{\partial z}{\partial x}\frac{\partial x}{\partial s} +
  \frac{\partial z}{\partial y}\frac{\partial y}{\partial s}\\
  \textrm{Implicit Diff: } \frac{\partial z}{\partial x} &=
  -\frac{F_x}{F_z}\\
  \textrm{Gradient: } \grad f &= <f_x,f_y,f_z>\\
  \textrm{Normal of Tangent Plane: }\vec n &= \grad {f}\\
  \textrm{Directional Derivative: } D_{\vec u} f &= \grad{f} \cdot \vec u \\
  \textrm{2nd Derivative Test: } D &= f_{xx}*f_{yy} - f_{xy}^2\\
  &\textrm{(a) } D > 0 \textrm{ and } f_{xx} > 0 \rightarrow f(a,b)
  \textrm{ is
    a local min}\\
  &\textrm{(b) } D > 0 \textrm{ and } f_{xx} < 0 \rightarrow f(a,b)
  \textrm{ is
    a local max}\\
  &\textrm{(c) } D < 0 \rightarrow f(a,b)
  \textrm{ is
    a saddle point}\\
  \textrm{Lagrange Multipliers: } \grad f &= \lambda \grad g + \mu
  \grad h\\
  g &= k\\
  h &= c
\end{align*}

Part 3
Multiple Integrals:
\begin{align*}
  \textrm{Double Integral: } \int\int_R&f(x,y)\\
  \textrm{Triple Integral: } \int\int\int_R&f(x,y,z)\\\\
  \int\int_D dA &= Area(D)\\
  \int\int\int_B dV &= Volume(B) \\
  \textrm{Interated Integral: }\int\int_R f(x,y) dA &= \int_a^b \int_c^d f(x,y) dydx = \int_c^d
  \int_a^b f(x,y) dxdy\\
  \textrm{Double Integral (Constant Bounds): }\int\int_R g(x)h(y)dA &=
  \int_a^b*g(x) dx \int_c^d h(y) dy&\\  
  \textrm{Iterated Integral: } \int\int\int_Bf(x,y,z) dV &=
  \int_r^s\int_c^d\int_a^b f(x,y,z)dxdydz \\
  \textrm{Triple Integral (Constant Bounds): }\int\int\int_B f(x)g(y)h(z)dV &= \int_a^b*f(x) dx \int_c^d g(y) dy \int_r^sh(z)dz\\  
\end{align*}
Change of Variable:
\begin{align*}
 \textrm{2D Jacobian: }J &= |\grad x(u,v) \times \grad y(u,v)| =
  \begin{Vmatrix}
    \grad x(u,v)\\
    \grad y(u,v)\\
  \end{Vmatrix}\\
  \textrm{3D Jacobian: } J &=   
   \begin{Vmatrix}
    \grad x(u,v,w)\\
    \grad y(u,v,w)\\
    \grad z(u,v,w)\\
  \end{Vmatrix}\\
  dA &= dydxdz = Jdudvdw\\
  \textrm{Polar: } 
  r^2 &= x^2+ y^2\\
  x &= r\cos{\theta}\\
  y &= r\sin{\theta}\\
  z &= z\\
  J &= r\\
  \textrm{Spherical: } 
  \rho^2 &= x^2+y^2+z^2\\
  x &=\rho\sin{\phi}\cos{\theta}\\
  y &=\rho\sin{\phi}\sin{\theta}\\
  z &=\rho\cos{\phi}\\
  J &= \rho^2\sin(\phi)
\end{align*}

Applications of Multiple Integrals
\begin{align*}
  \textrm{Mass: } m &= \int\int\int_E \rho(x,y,z)dV\\
  \textrm{Moment: } M_{vw}&= \int\int\int_Eu\rho(u,v,w)dV\\
  \textrm{Center of Mass Cord: }\bar u &= \frac{M_{vw}}{m}\\
  \textrm{Center of Mass: } c_m &= (\bar x,\bar y, \bar z)\\
\end{align*}
Part 4
Vector Fields
\begin{align*}
  \textrm{Line integral: }\\
  \int_C \vec F \cdot d \vec r &= \int_C F \cdot \vec T ds\\
  \int_C f(x,y)ds &=\int_a^bf(x(t),y(t))\sqrt{\frac{dx}{dt}^2+\frac{dy}{dt}^2}dt\\
  \int_C \vec F \cdot d\vec r &= \int_C P{dx}+Q{dy}+R{dz} \textrm{
    where  }\vec F = P \vec i + Q \vec j + R \vec k\\
  \textrm{Vector Operations 2: }\\
  \textrm{Curl: } curl \vec F &= \curl \vec F\\
  curl(\grad f) &= \curl \grad f= 0\\
  \textrm{Divergence: } div \vec F &= \div \vec F = \frac{\partial
    P}{\partial x}+\frac{\partial Q}{\partial y} + \frac{\partial
    R}{\partial z}\\
  div(curl (\vec F))&=\div\curl \vec F = 0\\
  \textrm{Parametric surface: } \vec \vec r(u,v) &= <x(u,v),y(u,v),z(u,v)>\\
  \textrm{Surface Area: }\\
  \vec N &= \vec r_u \times \vec r_v\\
  A(S) &= \int\int_D |\vec N|dA\\
  A(S) &= \int\int_D \sqrt{1 + z_x^2+z_y^2}dA\\
  \textrm{Surface Integrals: }\\
  \int\int f(x,y,z)dS &= f(\vec r(u,v))|\vec N|dA\\
  \int\int f(x,y,z)dS &= \int\int_D f(x,y,g(x,y))\sqrt{1 +
    z_x^2+z_y^2}dA\\
  \vec n &= \frac{N}{|N|}\\
  \int\int_S \vec F \cdot d\vec S &=\int\int_S \vec F \cdot \vec n
  dS\\ 
  \int\int_S \vec F \cdot d\vec S &=\int\int_D \vec F \cdot \vec N
  dA\\
  \int\int_S \vec F \cdot d\vec S &= \int\int_D(-Pg_x-Qg_y+R)dA\\
\end{align*}
Conservative Vector Fields
\begin{align*}
    \frac{\partial P}{\partial y} &= \frac{\partial Q}{\partial x}
  \textrm{   and D is an open simply-connected region}\\
  \vec F(x,y,z) &= \grad f\\
  \oint \vec F \cdot \vec dr &= 0\\
  \int_C\grad f \cdot d \vec r &= f(\vec r(b)) - f(\vec r(a))\\
   \textrm{Green's Theorem: }\\
   \int_C P dx+ Qdy &= \int\int_D(\frac{\partial Q}{\partial
     x}-\frac{\partial P}{\partial y})dA\\
   \oint_C\vec F \cdot d\vec r &= \int \int_D (\curl \vec F)\cdot \vec
   k dA\\
   \oint_C\vec F \cdot \vec n ds &= \int\int_D \div {\vec F(x,y)} dA\\
   \textrm{Stoke's Theorem}\\
   \int_C\vec F \cdot d\vec r &= \int\int_S \curl \vec F \cdot d \vec
   S\\
   \textrm{Divergence Theorem}\\
   \int\int_S \vec F \cdot d \vec S &= \int\int\int_E \div {\vec F} dV
\end{align*}
\end{document}