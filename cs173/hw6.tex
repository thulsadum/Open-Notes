\documentclass[12pt]{article}
\textwidth = 6.5 in
\textheight = 9 in
\oddsidemargin = 0.0 in
\evensidemargin = 0.0 in
\topmargin = 0.2 in
\headheight = 0.0 in
\headsep = 0.0 in
\parskip = 0.2in
\parindent = 0.0in

\usepackage{amsmath,amssymb,enumerate,gastex}

% =========================================================
\begin{document}

\begin{center}
{\LARGE
\textbf{CS 173: Discrete Structures, Spring 2011}
\\
\textbf{Homework 6}
\\[1ex]}

This homework contains 2 problems worth a total of 26 points.
It is due on Wednesday, March 16th 5pm.

\end{center}

\bigskip\hrule

\begin{enumerate}

%----------------------------------------------------------------------
\item \textbf{Graph Concepts [10 points]}

Answer the following questions for the graph below: 

\begin{enumerate}
\item (2 points) Find the size of the \textbf{minimum cut}, i.e. the smallest number of edges that you could remove to disconnect the graph into two separate connected components.  Briefly justify your answer.
\item (2 points) What is the diameter of the graph? List two nodes with the greatest distance.

\item (6 points) What is the chromatic number of the graph? Provide a proof.

    \begin{picture}(120,50)(0,0)
        \gasset{Nframe=n}  %no border around nodes
        \gasset{AHnb=0} % no arrowheads
%        \gasset{Nadjust=w} %match width to label

        \node(xxx)(10.0,45.0){G1:}

        \node(xb)(40.0,5.0){b}
        \node(xc)(85.0,5.0){c}
        \node(xa)(20.0,25.0){a}
        \node(xd)(85.0,25.0){d}
        \node(xe)(40.0,45.0){e}
        \node(xf)(85.0,45.0){f}
        \node(xg)(60.0,25.0){g}
        \node(xh)(65.0,5.0){h}

    \drawedge(xh,xb){}
   \drawedge(xh,xd){}
    \drawedge(xh,xc){}
    \drawedge(xa,xb){}
    \drawedge(xa,xe){}
    \drawedge(xc,xd){}
    \drawedge(xf,xd){}
    \drawedge(xf,xe){}
    \drawedge(xa,xg){}
    \drawedge(xe,xb){}
    
    \drawedge(xb,xd){}

    \drawedge(xb,xg){}
    \drawedge(xd,xg){}
    \drawedge(xe,xg){}

    \end{picture}
\end{enumerate}


%----------------------------------------------------------------------
\item \textbf{Induction [16 points]}

Prove the following propositions using induction.   Be sure to label the
main parts of your proof and state your inductive hypothesis clearly. 

\begin{enumerate}

\item (8 points) For $n \in \mathbb{N}, 5 | (n^{5} - n)$.  
(Hint: if you need to compute $(x+y)^n$, 
you can avoid multiplying it out by hand if you
look up the ``binomial theorem'' or ``Pascal's triangle.'')

\item (8 points) For $n \in \mathbb{N}, \displaystyle\sum\limits_{i=0}^n i \cdot i! = (n+1)! - 1$.
(Hint:   notice that $0!$ is defined to be 1.)

\end{enumerate}





\end{enumerate}

\end{document}
