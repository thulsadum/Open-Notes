\documentclass[12pt]{article}
\textwidth = 6.5 in
\textheight = 9 in
\oddsidemargin = 0.0 in
\evensidemargin = 0.0 in
\topmargin = 0.2 in
\headheight = 0.0 in
\headsep = 0.0 in
\parskip = 0.2in
\parindent = 0.0in

\usepackage{amsmath,amssymb,enumerate}

% =========================================================
\begin{document}

\begin{center}
{
\textbf{Marcell Vazquez-Chanlatte}
\\
\textbf{F2}
\\
\textbf{Group: Anthony Louie, Nathan Pantaleo, Eric Mills}
\\[1ex]}

\end{center}

\bigskip\hrule

\begin{enumerate}


%----------------------------------------------------------------------
\item \textbf{Euclidean algorithm [6 points]}

GCD(42823,6409)\\

\begin{tabular}{| 1 | c || r | }
 42823 & 6409 & 4369 \\
 6409 & 4369 & 2040 \\
 4369 & 2040 & 289 \\
 2040 & 289 & 17 \\
 289 & 17 & 0 \\
 17 & 0 & \\
\end{tabular}

$\gcd(42823,6409)=17$

%----------------------------------------------------------------------
\item \textbf{Sets [12 points]}

Suppose you were given the following sets:
\begin{eqnarray*}
\textbf{A} &=& \{\text{book}, \text{desk}, \text{pencil}\} \\
\textbf{B} &=& \{7,8,11\} \\
\textbf{C} &=& \{\text{book}, \text{desk}\} \\
\textbf{D} &=& \{(\text{desk},\text{pencil}), \text{cup}\} \\
\textbf{E} &=& \{6,13,21, 405\} \\
\end{eqnarray*}

List the elements in each of the following sets.  
\begin{enumerate}
\item $A - (D \cup C) = \{pencil\}$
\item $A \cup D = \{book,desk,pencil,(desk,pencil),cup\}$ 
\item $B \times C = \{(7,book),(8,book),(11,book),(7,desk),(8,desk),(11,desk)\}$
\item $\{x \in (B \cup E) \mid x \text{ is prime} \} = \{ 7,11,13\}$
\item $\emptyset \times (C \cap E) = \emptyset $
\end{enumerate}

Find the value of the following expression.  Show your work.

\begin{enumerate}
\item $|(A \cup C) \times (B \cup E)|$ 
\item $|A \cup C| = | \{book,desk,pencil \}| = 3$
\item $|B \cup E| = | \{ 7,8,11,6,13,21,405\}| = 7 $
\item $7*3 = 21$
\end{enumerate}

\newpage
%----------------------------------------------------------------------
\item \textbf{Equivalence Classes and Computation [10 points]}

\begin{enumerate}[(a)]

\item (2 points) Suppose that we treat integers as equivalent if they
are congruent mod $13$.  
List the three positive and three negative
members of the equivalence class $[-3]$ that are smallest in absolute
value.\\
$(-x-3)*k=13$\\
$\{-29,-16,-3,10,23,36\}$

\item (4 points) In $\mathbb{Z}_{11}$ (i.e. using arithmetic modulo $11$), compute
the values of $[7]^k$ for $k = 1, 2, \ldots, 6$.
Express your final answer as $[x]$, where $0 \le x < 10$.   

\begin{tabular}{ 1 | c || r | }
\hline
 K & x      & [x] \\
 1 & 7      & [7] \\
 2 & 49     & [5] \\
 3 & 343    & [2] \\
 4 & 2401   & [3] \\
 5 & 16807  & [10] \\
 6 & 117649 & [4] \\
\end{tabular}

\item (4 points) Construct a multiplication table for the equivalence classes in
$\mathbb{Z}_5$.  The format of your table should be similar to the
multiplication tables given in the number theory lecture notes.
In particular, the outputs in your table should have the form
$[x]$, where $0 \le x < 5$.

\begin{tabular}{ 1 | c | r | a | b | c }
\hline
   &  0  &  1  &  2  &  3  &  4  \\
 0 & [0] & [0] & [0] & [0] & [0] \\
 1 & [0] & [1] & [2] & [3] & [4] \\
 2 & [0] & [2] & [4] & [1] & [3] \\
 3 & [0] & [3] & [1] & [4] & [2] \\
 4 & [0] & [4] & [3] & [2] & [1] \\
\end{tabular}

\end{enumerate}

\newpage
%----------------------------------------------------------------------
\item \textbf{Proof with Primes [8 points]}

Prove that any prime number $p$ of the form $p=3k+1$ for an integer $k$
must also have the form $p=6k'+1$ for an integer $k'$.


\begin{quote}
Def 0: $a$ is odd iff $a=2b+1$ $a,b\in\mathbb{Z}$\\
Def 1: $a$ is even iff $a=2b$   $a,b\in\mathbb{Z}$\\
Def 2: $a$ is prime iff $a$ is only divisible by itself and $1$\\
Claim $\forall p,k \in \mathbb{Z} :$ p is prime\\
if $p=3k+1 $ then $p = 6k'+1 $
\end{quote}

\begin{enumerate}
\item Let $p = 3k+1$ where $p,k\in\mathbb{Z}: p$ is prime
\item Note $k > 0$ because $p \leq 1$ is not prime
\item Similarly $k \neq 1$ because $p=4$ is not prime
\item Observe that because $k$ is an integer $p\neq2$ (because $2<4$ and $p$ is discrete)
\item Also note that all primes $>$ $2$ are not divisble by $2$ Def $2$
\item Thus $p$ must be odd
\item Suppose $k$ is odd, then $k=2k'+1,$  $k'\in\mathbb{Z}$ Def $0$
\item Thus $3k=6k'+3$ , $3k+1=6k+4=2(3k+2) = p$ which is even by Def 0 -Contradiction
\item Thus instead consider $k$ is even $\rightarrow$ $k=2k'$ Def 1
\item Substituting $2k'$ for $k$ into $p=3k+1$
\item $p=6k'+1$ $\square$
\end{enumerate}
\newpage
%----------------------------------------------------------------------
\item \textbf{Direct Proof with Congruence mod k [10 points]}

There are several equivalent ways to define congruence mod $k$.
For this problem, use the following definition:

\textit{Def 3: For any integers $a$ and $b$ and any positive integer $k$,
$a \equiv b \pmod{k}$ if there is an integer $m$ such that $a = b + mk$.}

Prove following 

\begin{quote}
Def 4: $a|b$ iff , $b = ak$  $ b,a,k \in \mathbb{Z}$
Claim: For all integers $a$, $b$, $r$, and $s$, and all positive
integers $k$ and $j$,
if $a \equiv r \pmod{k}$ and $b \equiv s \pmod{j}$, then
$a^2 + b \equiv r^2 + s \pmod{d}$ where $d = \gcd(k,j)$.
\end{quote}

\begin{enumerate}
\item Let $a \equiv r \mod{k}$ and $b \equiv s\mod{j}$ where $k \wedge j > 0$  $a,r,k,b,k,j \in \mathbb{Z}$
\item Thus $a=r+mk$ and $b=s+nj$  $n,m\in\mathbb{Z}$ Def $3$
\item Note $a^2=r^2+2rmk+m^2k^2$
\item Adding $a^2$ to $b$ results in $a^2+b=(r^2+s)+(2rmk+m^2k^2+nj)$
\item Let $l=(2rmk+m^2k^2+nj)$  $l \in \mathbb{Z}$
\item Substituting $a^2+b=(r^2+s) +l$
\item Let $d=\gcd(k,j)$  $d \in \mathbb{Z}$
\item Thus $k=dh$ and $j=dg$  $g,h \in \mathbb{Z}$ Def $4$
\item Substituting $dh$ and $dg$  for $k$ and $j$ respectively $l=2drmh+d^2m^2h^2+dng = d(2rmh+dm^2h^2+ng)$
\item Let $f=(2rmh+dm^2h^2+ng)$ $f\in\mathbb{Z}$
\item Substituting $a^2+b= (r^2+s) +df$ $\square$
\end{enumerate}

\end{enumerate}
\end{document}
