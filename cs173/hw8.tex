\documentclass[12pt]{article}
\textwidth = 6.7 in
\textheight = 9.2 in
\oddsidemargin = 0.0 in
\evensidemargin = 0.0 in
\topmargin = 0.2 in
\headheight = 0.0 in
\headsep = 0.0 in
\parskip = 0.2in
\parindent = 0.0in

\usepackage{amsmath,amssymb,enumerate,enumerate}
%\def\rmdefault{bch} % Use Charter for main text font.

% =========================================================
\begin{document}

\begin{center}
{\LARGE
\textbf{Marcell Vazquez-Chanlatte}
\\
[1ex]}
\end{center}
\begin{center}
\textbf{Eric Mills, Anthony Louie}
\\
\textbf{F2}  
\end{center}


\begin{enumerate}

%------------------------------------------------------------
\item
\textbf{Tree induction [13 points]}

Consider the following grammar $G$ with four rules.

\begin{eqnarray*}
S &\rightarrow& a\ S \ b  \ \ \mid \ \ S\ c\ a \\
S &\rightarrow& a\ b \ \ \mid\ \  c\ a
\end{eqnarray*}


\begin{enumerate}[(a)]

\item Draw two other
trees matching grammar $G$ with height  $\le 4$.
\item Draw two trees which do not match grammar $G$.
These trees must have height $\le 4$ and must have all
nodes labelled with one of $S$, $a$, $b$, or $c$.
\newpage
\item Claim: In a tree matching grammar
$G$, the number of $a$ nodes equals the sum of the numbers of $b$ and
$c$ nodes.

Base:
\begin{enumerate}
\item Suppose a tree has height $1$ and matches grammar $G$.
\item Clearly the only two grammars that can be produces are $S
  \rightarrow ab | ca$
\item Case $S \rightarrow ab$
\item Suppose the tree has a root $S$ and two children $a$ and $b$
\item Observe that the number of $a = 1, b = 1,c = 0$ thus $a = 1 = 1 + 0
  =  b + c : a,b,c \in \mathbb{Z}$
\item Thus this tree matches grammar $G$
\item Case $S \rightarrow ac$
\item See Case $S \rightarrow ab$ but exchange $c$ for $b$

\end{enumerate}

Induction of height
\begin{enumerate}
\item I.H. Suppose that for a tree of height $h$ that matches grammar
  $G$, the number of $a$ nodes equals the sum of the numbers of $b$
  and $c$ nodes
\item Consider a tree of height $h+1$ that matches grammar $G$
\item Notice that $S \rightarrow aSb|Sac$ is the same as  $S
  \rightarrow ab|ac$ with the addition of $S$ thus the counting of
  $a,b,c$ is the same as in the Base case for these.
\item Also Note that $S \rightarrow aSb|Sac$ produced a new row thus
  cannot be applyed to the terminal row
\item Applying the IH, for sub tree of all nodes up to height $h$, the number of $a$ nodes equals the sum of the numbers of $b$
\item Also observe that leafs of on row $h+1$ cannot contain $S$, and
  thus can only be produced by $S \rightarrow ab|ac$
\item Applying the base cases, the number of $a$ nodes on this row
  equals the sum of the numbers of $b$ on this row
\item Thus the number of $a$ nodes equals the sum of the numbers of
  $b$ for the whole tree by closure $\square$
\end{enumerate}



\end{enumerate}

\newpage
%----------------------------------------------------------------------

\item
\textbf{Big-O proofs [8 points]}

\begin{enumerate}[(a)]
\item Claim: $5x^2$ is $O(x^3 - 17x)$.
  \begin{enumerate}
  \item Consider $c = 10, k = 1000$ for $c,k \in \mathbb{Z}$
  \item Consider $x \geq k$ for $x \in \mathbb{N}$
  \item Observe that $0<5x^2$
  \item Also Observe that $5x^2 < 10x^2$ 
  \item Notice that $17x+x^2 = x(17+x) \leq x^3$
  \item Thus $x^2 \leq x^3 - 17x$ and $10x^2 \leq 10(x^3 - 17x)$
  \item Therefore, $0 \leq 5x^2 \leq 10x^2 \leq 10(x^3-17x) =
    c*(x^3-17x)$
  \item Thus by applying the Def of Big O, $5x^2$ is $O(x^3 - 17x)$.
  \item $\square$
  \end{enumerate}
\item Claim: $3^x$ is {\bf not} $O(2^x)$.
  \begin{enumerate}
  \item Proof by Contradiction
  \item Suppose $3^x$ is $O(2^x)$
  \item Applying the Def of Big O, $0 \leq 3^x \leq c2^x$ for $x \geq
    k$    $k,c \in \mathbb{Z}^+$
  \item Therefore, $0 \leq \frac{3^x}{2^x} \leq c$
  \item Notice however that $\frac{3^x}{2^x}$ is unbounded, i.e. it
    can be greater than $c$ - Contradiction
  \item Thus $3^x$ is {\bf not} $O(2^x) \#$
  \end{enumerate}
\end{enumerate}

\newpage

%----------------------------------------------------------------------

\item 
\textbf{Unrolling}

\begin{enumerate}
\item $T(1) = 1$
\item $T(n) = 3\cdot T(n/2) + 7 \hspace*{1.0in} \text{ for }n \ge 2$
\item $T(n) = 3(3*T(n/4) +7) +7$
\item ..   = $3(3*3*(3*T(n/8) + 7)+ 7 ) +7$
\item ..   = $3^3*T(n/2^3) + 3^2*7+3^1*7+3^0*7$
\item ..   = $3^3*T(n/2^3) + 7*\displaystyle\sum\limits_{i=0}^2 (3^i)$
\item  K'th step
\item .. = $3^k*T(n/2^k) + 7*\displaystyle\sum\limits_{i=0}^{k-1} (3^i)$
\item .. = $3^k*T(n/2^k) + 7*\displaystyle\sum\limits_{i=0}^{k-1} (3^i)$
\item .. = $3^k*T(n/2^k) + 7*\frac{1}{2}(3^k-1)$
\item .. = $3^{\log(n)}*T(n/2^{\log(n)}) + 7*\frac{1}{2}(3^{\log(n)}-1)$
\item .. = $3^{\log(n)} + \frac{7}{2}(3^{\log(n)})-\frac{7}{2}$
\item .. = $3^{\log(n)}(1+\frac{7}{2}) -\frac{7}{2}$
\item .. = $3^{\log(n)}(\frac{9}{2}) -\frac{7}{2}$
\item .. = $\frac{3^{\log(n)+2}-7}{2}$
\end{enumerate}


%----------------------------------------------------------------------

\end{enumerate}

\end{document}
