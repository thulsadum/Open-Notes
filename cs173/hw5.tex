\documentclass[12pt]{article}
\textwidth = 6.5 in
\textheight = 9 in
\oddsidemargin = 0.0 in
\evensidemargin = 0.0 in
\topmargin = 0.2 in
\headheight = 0.0 in
\headsep = 0.0 in
\parskip = 0.2in
\parindent = 0.0in

\usepackage{amsmath,amssymb,enumerate,gastex}

% =========================================================
\begin{document}

\begin{center}
{\LARGE
\textbf{Marcell Vazquez-Chanlatte}
\\
\textbf{3/9/11}
\\
\textbf{Eric Mills, Anthony Louie, Nathan}
\\[1ex]}

\end{center}

\bigskip\hrule

\begin{enumerate}


%----------------------------------------------------------------------
\item \textbf{One-to-one}

\begin{quote}
 \item a. $f: \mathbb{R} \times \mathbb{R} \rightarrow \mathbb{R}$ such that $f(x,y) = x^5-y^2+8$
\end{quote}

 Not $1-1$ because $y^2$ is an even function thus $y$ and $-y$ produce the same result.\\
 For example $f(x,y) = x^2 - y^2 + 8 = x^2 - (-y)^2 + 8 = f(x,-y)$

\begin{quote}
 \item b. $f: \mathbb{R}^+ \rightarrow \mathbb{R}$ such that $f(x) = 2e^{-2x}$
  where $e$ is the base of the natural logarithm.   (Hint:  if I substituted
  another positive number for $e$, would anything important change?)
\end{quote}

 $1-1$\\
 Let $f(x) = f(y)$
 Thus $2e^{-2x} = 2e^{-2y}$
 Therefore $-2x = -2y$, $x=y$....
 Also the function is continuously decaying....
\begin{quote}
 \item c. $f: \mathbb{N} \times \mathbb{N} \rightarrow \mathbb{Z}$ such that $f(x,y) = \left|x+1\right|-\left|y^3\right|$
\end{quote}

 Not $1-1$\\
 Counter-example: $(8,2)$ and $(0,0)$ produce the same result. $|8+1| - |2^3| = 1 = |1| -|0^3|$
\newpage
%----------------------------------------------------------------------
\item \textbf{Function proofs}


\begin{quote}
\item Let's define a function $f$ as follows:
\item $f: [1/4,3] \rightarrow [0,1] $ such that $f(x) = \frac{4x-1}{2x+5}$
\item Using the definitions of one-to-one and onto, prove that $f$ is a
bijection. 
\end{quote}

\begin{enumerate}
\item Let $f: [\frac{1}{4},3] \rightarrow [0,1]$ s.t $f(x) = \frac{4x-1}{2x+5}$
\item Note that bijective means onto and $1-1$
\item Claim 1: $f$ is onto:
\item Note the $f$ is only undefined at $x= -2.5$ $,x\in \mathbb{R}$
\item Also note that $-2.5$ isn't in the interval $[\frac{1}{4}, 3]$
\item Consider $x= \frac{5y+1}{2*(2-y)}$ where $y$ is an arbitrary element  $\in [0,1]$ $,x,y\in \mathbb{R}$
\item Let $a = 4x-1$ and $b = 2x+5$
\item $f(x) = \frac{a}{b}$
\item Subsituting $x$ in $a$ and $b$
\item $a = \frac{11y}{2-y}$ and $b = \frac{11}{2-y}$
\item Thus $f(x) = \frac{a}{b}  = \frac{11y}{11} = y$ 
\item Thus $\forall y \in [0,1], \exists x \in[\frac{1}{4}, 3]$
\item Thus $f$ is onto
\item Claim 2: $f$ is $1-1$
\item Let $f(x) = f(y) x,y \in \mathbb{R}$
\item Notice that then $\frac{4y-1}{2y+4} = \frac{4x-1}{2x+4}$
\item Simplifying $(2x+4)(4y-1) = (4x-1)(2y+4), 8xy-2x+16y-4=8xy-2y+16x-4$
\item $-2x+16y = -2y+16x, 18y=18x, x=y$
\item Thus is $1-1$
\item Therefore, because $f$ is onto and $1-1$ it is bijective
\item $\square$
\end{enumerate}
\newpage
%----------------------------------------------------------------------
\item \textbf{Pigeonhole principle}

\begin{quote}
\item Consider an equilateral triangle with side equal 1 in. What is the
maximum number of points you can place in that triangle such that no
two points are $\le 0.5$ in apart?   (Hint:  try dividing the triangle
into smaller regions and use the pigeonhole principle.)
\end{quote}

\begin{enumerate}
\item Note that the length of each side is $1$ therefore base, $b =1$ $b \in \mathbb{R}$
\item Also note that the height, $h = .5*\sqrt{3}$ $h \in \mathbb{R}$
\item Therefore the Area of the triangle, $A = .5*b*h = .25*\sqrt{3}$
\item The minimum Area of the Circle/triangle intersection created by a point P is at the vertices
\item Thus because the angle is $\frac{\pi}{3}$ which is $\frac{1}{6}$ of the Circle's Area
\item $P_{Area} = \frac{\pi}{24}$
\item Note that $3*P_{Area} = \frac{\pi}{8}$ and $A-3*P = \frac{2*\sqrt{3}-\pi}{8} < \frac{pi}{24}$
\item Therefore only one more region can be created.
\item Let $R$ be the set of regions and $C$ be the set of Points then s.t $|R|=4$
\item Thus by P.P if $|C| > 4$ two point will occupy the same region thus violating the terms
\item Therfore max = $4$
\end{enumerate}
\newpage
%----------------------------------------------------------------------
\item \textbf{Counting and graphs}

\begin{quote}
Suppose that $V$ is a set with $n$ elements.   How
many different possible (simple, undirected) 
graphs are there with vertex set $V$?  Briefly justify your answer.
\end{quote}

\begin{enumerate}
\item Observe that a complete graph a $n$ Vertices, $n =|V|$ has $k$ edges where $k = \frac{n*(n-1)}{2}$
\item Each edge has two states enabled or disabled.
\item Thus a binary tree of $k$ levels is produced modeled by $2^k$
\end{enumerate}

%----------------------------------------------------------------------
\item \textbf{Cycles and isomorphism}

For the two graphs below, answer the following questions.
Justify your answers.

\begin{quote}
\item A. How many isomorphisms are there from G1 to itself?
\end{quote}


There are a few ways two derive this...Looking at G1 as a perfectly symetrical shape, there are $4$ axies of symmetry, and looking counting rotations there are $6$. Thus $10$ isomorphisms//
Instead focusing on individual nodes... we come up with the following possibilites:
\begin{enumerate}
\item $= 2$ choices
\item $= 1$ choices
\item $= 2$ choice
\item $= 1$ choice
\item $= 2$ choice
\item $= 1$ choice
\item $= 1$ choice
\end{enumerate}

Therefore there are $10$ isomorphisms...which matches our previous conclusion

\begin{quote}
\item B. Are G1 and G2 isomorphic?
\end{quote}

Not Isomorphic because the subgraph in $G2$ where a 2 degree is connected to a 3 degree which is connected to another 2 degree is missing in $G1$

\newpage
\begin{quote}
\item C. For which values of $n$ does $G1$ contain a cycle
of length $n$?
\end{quote}

$n = 3$ $bc,cg,gb$\\
$n = 4$ $ab,bg,ge,ea$\\
$n = 5$ $ab,bc,cg,ge,ea$\\
$n = 6$ $ab,bc,cd,df,fe,ea$\\
$n = 7$ $ab,bg,gc,cd,df,fe,ea$\\


    \begin{picture}(120,50)(0,0)
        \gasset{Nframe=n}  %no border around nodes
        \gasset{AHnb=0} % no arrowheads
%        \gasset{Nadjust=w} %match width to label

        \node(xxx)(10.0,45.0){G1:}

        \node(xb)(40.0,5.0){b}
        \node(xc)(60.0,5.0){c}
        \node(xa)(20.0,25.0){a}
        \node(xd)(80.0,25.0){d}
        \node(xe)(40.0,45.0){e}
        \node(xf)(60.0,45.0){f}
        \node(xhub)(60.0,25.0){g}

    \drawedge(xc,xb){}
    \drawedge(xa,xb){}
    \drawedge(xa,xe){}
    \drawedge(xc,xd){}
    \drawedge(xf,xd){}
    \drawedge(xf,xe){}

    \drawedge(xb,xhub){}
v    \drawedge(xc,xhub){}
    \drawedge(xe,xhub){}
    \drawedge(xf,xhub){}

    %======================

        \node(yyy)(90.0,45.0){G2:}

        \node(yb)(130.0,15.0){1}
        \node(yc)(120.0,5.0){2}
        \node(ya)(140.0,25.0){3}
        \node(yd)(160.0,25.0){4}
        \node(ye)(130.0,35.0){5}
        \node(yf)(120.0,45.0){7}
        \node(yhub)(100.0,25.0){6}

    \drawedge(yc,yb){}
    \drawedge(ya,yb){}
    \drawedge(ya,ye){}
    \drawedge(yc,yd){}
    \drawedge(yf,yd){}
    \drawedge(yf,ye){}

    \drawedge(ya,yhub){}
    \drawedge(yc,yhub){}
    \drawedge(ye,yhub){}
    \drawedge(yf,yhub){}


    \end{picture}





\end{enumerate}

\end{document}
