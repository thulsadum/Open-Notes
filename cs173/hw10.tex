\documentclass[12pt]{article}
\textwidth = 6.7 in
\textheight = 9.2 in
\oddsidemargin = 0.0 in
\evensidemargin = 0.0 in
\topmargin = 0.2 in
\headheight = 0.0 in
\headsep = 0.0 in
\parskip = 0.2in
\parindent = 0.0in

\usepackage{amsmath,amssymb,enumerate,enumerate,gastex}
% \def\rmdefault{bch} % Use Charter for main text font.

% =========================================================
\begin{document}

\begin{center}
  {\LARGE
  \textbf{Marcell Vazquez-Chanlatte}
  \\
  \textbf{F2}
  \\[1ex]}
\end{center}

\begin{center}
  Eric Mills, Anthony Louie
\end{center}

\begin{enumerate}

  % ------------------------------------------------------------
\item
\textbf{Set-valued functions}

 $f:\mathbb{N} \rightarrow \mathbb{P}(\mathbb{N})$, $f(n) = \{p \in \mathbb{N} : n | p\}$.

\begin{enumerate}
\item $f(m)= f(p) \cap f(q)$\\
  Notice that $q$ can be thought of as a multiple of $p$. Therefore only
  the elements of $f(p)$ that have a multiple of a $q$ as the multiple
  of $p$ are selected. This happens to be defined as an intersection.

\item Observe that this is merely an extension of part $a$ with
  multiple sets. Thus the same logic applys recursively.\\
\\
Lets define  $T:\mathbb{N} \rightarrow \mathbb{P}(\mathbb{N})$ to be\\
$T(1) = f(p_1)$\\
$T(n) = f(p_n) \cap T(n-1)$\\

Therefore $f(m) = T(m)$
\item Notice that if $f(p)$ is a subset of $f(q)$ then all the
  multiples of $p$ are in $f(q)$.\\
  Consider $p=q*k$ for some $p,q,k\in \mathbb{N}$\\
  $f(q)$ would then contain multiples of $p$.\\
  Therefore $f(p)$ is a subset of $f(q)$ for $p = q*k$
\item $m=p*q$ see part a for reasoning.
\end{enumerate}


\newpage
%------------------------------------------------------------
\item
\textbf{Set-valued functions and partitions}

\begin{enumerate}
\item  $M:V\times\mathbb{N} \rightarrow \mathbb{P}(V)$ by $M(x,n) =\{y \in V \mid \text{ there is a path of length } n\text{ from }x\text{ to }y \}$
\begin{tabular}{| 1 || r | }
  $M(c,0)$ & $\{c\}$         \\
  $M(c,1)$ & $\{a,d,f,k\}$   \\
  $M(c,2)$ & $\{e,b,f,h,k\}$ \\
  $M(c,3)$ & $\{e,b,g,h\}$   \\
  $M(c,4)$ & $\{a,b,d,g,h\}$ \\ 
  $M(c,5)$ & $\{b,d,e,f,h\}$ \\
\end{tabular}
\item No, because if a node is reachable using to diferent path
  lengths, to instances of that node will show up in the
  set....violating the def of a Partition.
\item 

  \begin{itemize}
  \item[] $Q(x,0) = \{x\}$
  \item[] For any $n \ge 1$, $Q(x,n) = \{y \in V \mid y \text{ is a neighbor of } p 
    \text{ for some } p \in Q(x,n-1)\}$
  \end{itemize}

  They are not equal because $Q(x,n)$ generates a walk instead of a
  path (Due to not blocking previously visited nodes). Therefore it may
  revisit a node which $P(x,n)$ clearly cannot.
  
\end{enumerate}







% ================================================================

\end{enumerate}
\end{document}
